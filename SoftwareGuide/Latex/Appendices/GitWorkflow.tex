\chapter{ITK Git Workflow}
\label{ch:ITKGitWorkflow}

This chapter describes the workflow adopted by the ITK community to develop
the software. The adopted Git-based \textbf{branchy workflow} is an efficient
and flexible model to develop modern software.


\section{Git Setup}
\label{sec:Git Setup}

Visit the main \href{https://www.git-scm.com/download}{Git download site}, and
depending on your operating system, follow the guidelines.


\subsection{Windows}
\label{subsec:Windows}

Git comes in two flavors on Windows:

\begin{itemize}
\item A Windows native application installer
\item A Cygwin package
\end{itemize}

Choose one and stick with it. They do not get along well in a given work tree
on disk (the repository formats are compatible but the ``stat cache'' of the
work tree is not unless \code{core.filemode} is false).


\subsubsection{Git for Windows}
\label{subsubsec:GitForWindows}

Download the ``git for windows'' executable from the
\href{https://git-for-windows.github.io/}{git for windows} site. You want to
download the file that is named something like

\begin{minted}[baselinestretch=1,fontsize=\footnotesize,linenos=false,bgcolor=ltgray]{bash}
Git-2.14.2.2-64-bit.exe
\end{minted}

Note that the filename changes as new versions are released.

Run the installer. When prompted, choose to not modify the \code{PATH} and
choose the \code{core.autocrlf=true} option. Launch the \code{Git Bash} tool to
get a command line shell with Git.


\subsubsection{Cygwin}
\label{subsubsec:Cygwin}

Install the following packages:
\begin{itemize}
\item \textbf{git}: Git command-line tool
\item \textbf{gitk}: Graphical history browser
\item \textbf{git-completion}: Bash shell completion rules
\end{itemize}

Launch a Cygwin command prompt to get a command line shell with Git.


\subsection{macOS}
\label{subsec:macOS}

\subsubsection{Xcode 4}
\label{subsubsec:Xcode4}

If you have Xcode 4 installed, you already have git installed.

Verify with:
\begin{minted}[baselinestretch=1,fontsize=\footnotesize,linenos=false,bgcolor=ltgray]{bash}
which git
/usr/bin/git
\end{minted}

\begin{minted}[baselinestretch=1,fontsize=\footnotesize,linenos=false,bgcolor=ltgray]{bash}
git --version
git version 1.7.4.4
\end{minted}


\subsubsection{OS X Installer}
\label{subsubsec:OSXInstaller}

Download an installer from
\href{https://code.google.com/archive/p/git-osx-installer/}{code.google.com}.


\subsubsection{MacPorts}
\label{subsubsec:MacPorts}

Enter these commands:

\begin{minted}[baselinestretch=1,fontsize=\footnotesize,linenos=false,bgcolor=ltgray]{bash}
sudo port selfupdate
sudo port install git-core +doc
\end{minted}


\subsection{Linux}
\label{subsec:Linux}

Popular Linux distributions already come with packages for Git. Typically the packages are called:
\begin{itemize}
\item \textbf{git-core}: Git command-line tool
\item \textbf{git-doc}: Git documentation
\item \textbf{gitk}: Graphical history browser
\end{itemize}


\section{Workflow}
\label{sec:Workflow}

\subsection{A Primer}
\label{subsec:APrimer}

This primer details a possible workflow for using Git and ITK. There are many
ways to use Git and it is a very flexible tool. This page details my particular
way of working with Git, and is certainly not the last word. It is also
rambling collection of tips and experiences that I've picked up in the last
few years of using Git.

It is worth trying to explain some high-level Git concepts. A
\href{https://carthik.net/blog/vault/2007/08/21/its-a-feature-not-a-bug/}{feature}
(or a
\href{https://blog.codinghorror.com/thats-not-a-bug-its-a-feature-request/}{bug})
that sets Git apart from Subversion is its distributed nature. In practice,
that means that ITK needs to ``bless'' a repository for it to be the
``official'' source of ITK. This has already been done at the
\href{https://github.com/InsightSoftwareConsortium/ITK}{ITK GitHub repository}.

Another Git concept is that of a \textbf{commit}. Git uses a
\href{https://en.wikipedia.org/wiki/SHA-1}{SHA1 hash} to uniquely identify a
change set. The hash is (almost) guaranteed to be unique across your project,
and even across all projects everywhere and for all time.

Quoting from the excellent Pro Git
book\footnote{\url{https://git-scm.com/book/en/v2}}:
\begin{quote}
A lot of people become concerned at some point that they will, by random
happenstance, have two objects in their repository that hash to the same SHA-1
value. What then?

If you do happen to commit an object that hashes to the same SHA-1 value as a
previous object in your repository, Git will see the previous object already in
your Git database and assume it was already written. If you try to check out
that object again at some point, you'll always get the data of the first
object.

However, you should be aware of how ridiculously unlikely this scenario is. The
SHA-1 digest is 20 bytes or 160 bits. The number of randomly hashed objects
needed to ensure a 50\% probability of a single collision is about $2^{80}$ (the
formula for determining collision probability is
$p = (n(n-1)/2) * (1/2^{160}))$. $2^{80}$ is $1.2 \times 10^{24}$ or 1 million
billion billion. That's 1,200 times the number of grains of sand on the earth.

Here's an example to give you an idea of what it would take to get a SHA-1
collision. If all 6.5 billion humans on Earth were programming, and every
second, each one was producing code that was the equivalent of the entire
Linux kernel history (1 million Git objects) and pushing it into one enormous
Git repository, it would take 5 years until that repository contained enough
objects to have a 50\% probability of a single SHA-1 object collision. A higher
probability exists that every member of your programming team will be attacked
and killed by wolves in unrelated incidents on the same night.
\end{quote}


\subsection{A Topic}
\label{subsec:ATopic}

This workflow is based on the branchy development workflow documented by
\href{https://git-scm.com/docs/gitworkflows}{Git help workflows}.


\subsubsection{Motivation}
\label{subsubsec:Motivation}

The primary goal of this workflow is to make release preparation and maintenance
easier. We set the following requirements, of which some are themselves
worthwhile goals:

\begin{itemize}
\item Amortize the effort of release preparation throughout development.
\item Support granular selection of features for release.
\item Allow immature features to be published without delaying release.
\item \textbf{Keep unrelated development paths (topics) independent of one
another}.
\item Maintain a clean shape of history (see Section~\ref{par:HistoryShape} on
page~\pageref{par:HistoryShape}).
\end{itemize}


\subsubsection{Design}
\label{subsubsec:Design}

The design of this workflow is based on the observation that meeting the
highlighted goal makes the other goals easy. It is based on branchy development
in which each branch has a well-defined purpose.

We define two branch types:
\begin{itemize}
\item \textbf{Topic Branch}
\begin{itemize}
\item Commits represent \textbf{changes} (real work)
\item Distinguished by \textbf{feature} (one topic per feature or fix)
\item Named locally by each developer (describe purpose of work)
\item Heads not published (no named branch on server)
\end{itemize}
\item Integration Branch
\begin{itemize}
\item Commits represent \textbf{merges} (merge topics together)
\item Distinguished by \textbf{stability} (release maintenance, release
preparation, development edge)
\item Named everywhere
\item Heads published on server
\end{itemize}
\end{itemize}


\subsubsection{Notation}
\label{subsubsec:Notation}

This chapter uses Git Directed Acyclic Graphs (DAG) to depict commit history:


\begin{table}
\begin{center}
\begin{tabular}{ m{0.3\textwidth} | m{0.6\textwidth} }
\toprule
\textbf{Meaning} & \textbf{Symbol} \\
\midrule
Branch name &
\begin{tikzpicture}
\node[DAGref,fill=pink](master){master};
\end{tikzpicture} \\
Current branch &
\begin{tikzpicture}
\node[DAGref,fill=pink](master){*master};
\end{tikzpicture} \\
Commit with parent in same branch &
\begin{tikzpicture}
\gitDAG[grow right sep = 2em, branch down = 3em]{
  {[nodes=placeholder commits] "..."} -- C1
};
\end{tikzpicture} \\
Commit with two parents (merge) &
\begin{tikzpicture}
\gitDAG[grow right sep = 1em, branch down = 3em]{
  {[nodes=placeholder commits] "..."} -- C1,
  C2; {C2 -- C1}
};
\end{tikzpicture} \\
%\bottomrule
\end{tabular}
\end{center}
\caption[Git DAG notation]{Git graphs notation.
\label{tab:GitDAGNotation}}
\end{table}

Topic branches generally consist of a linear sequence of commits forked off an
integration branch:

\begin{figure}
\centering
\begin{tikzpicture}
\gitDAG[grow right sep = 1em, branch down = 3em]{
  C1 -- {
    "" [placeholder];
    C2 -- C3 -- C4 -- C5
  }
};
\gittag
{master}       % node name and text
{right=of C1} % node placement
{C1}          % target
\gittag
{topic}       % node name and text
{right=of C5} % node placement
{C5}          % target
\end{tikzpicture}
\itkcaption{Topic branch.}
\label{fig:TopicBranch}
\end{figure}

Integration branches generally consist of a sequence of merge commits:
\begin{figure}
\centering
\begin{tikzpicture}
\gitDAG[grow right sep = 1em, branch down = 3em]{
  {C6 -- } -!- {C7 -- },
  C1 -- {C2 -- C3 -- C4 -- C5},
  {C8 -- C2} -!- {C9 -- C4};
  {C6 -- C3}; {C7 -- C5}
};
\gittag
{master}       % node name and text
{right=of C5} % node placement
{C5}          % target
\end{tikzpicture}
\itkcaption{Merge commits into the \code{master} branch.}
\label{fig:MergeCommitsInIntegrationBranch}
\end{figure}


\subsubsection{Published Branches}
\label{subsubsec:PublishedBranches}

We publish an \textit{integration} branch for each stage of development:
\begin{itemize}
\item \textbf{release}: Release maintenance (high stability). Only bug fixes
should be published here. Only the release manager can push here.
\item \textbf{master}: Release preparation (medium stability). Only mature
features and bug fixes should be published here.
\end{itemize}

Topic branches are not published directly; their names exist only in each
developer's local repositories.


\subsubsection{Development}
\label{subsubsec:Development}

We cover below the steps to take during each phase of development.

\paragraph{Initial Setup}
\label{par:InitialSetup}

These instructions generally provide all arguments to \code{git push} commands.
Some people prefer to use \code{git push} with no additional arguments to push
the current tracking branch. Run the command

\begin{minted}[baselinestretch=1,fontsize=\footnotesize,linenos=false,bgcolor=ltgray]{bash}
git config --global push.default tracking
\end{minted}

to establish this behavior. See the
\href{https://git-scm.com/docs/git-config}{git config} man-page for
details.


\paragraph{New Topic}
\label{par:NewTopic}

Create a new topic branch for each separate feature or bug fix. Always start the
topic from a stable integration branch, usually \textbf{master}. If the topic
fixes a bug in the current release, use \textbf{release}. In the following
section we review the steps and commands to create, develop, and publish a topic
branch based on \textbf{master}.

Update \textbf{master} to base work on the most recently integrated features.

\begin{minted}[baselinestretch=1,fontsize=\footnotesize,linenos=false,bgcolor=ltgray]{bash}
git checkout master
\end{minted}

\begin{figure}
\centering
\begin{tikzpicture}
\gitDAG[grow right sep = 1em, branch down = 3em]{
  -- C1
};
\gittag
{master}       % node name and text
{right=of C1} % node placement
{C1}          % target
\end{tikzpicture}
\itkcaption{Locate at \code{master}.}
\label{fig:CheckoutBranch}
\end{figure}

\begin{minted}[baselinestretch=1,fontsize=\footnotesize,linenos=false,bgcolor=ltgray]{bash}
git pull
\end{minted}

\begin{figure}
\centering
\begin{tikzpicture}
\gitDAG[grow right sep = 1em, branch down = 3em]{
  -- C1 -- C2
};
\gittag
{master}       % node name and text
{right=of C2} % node placement
{C2}          % target
\end{tikzpicture}
\itkcaption{Bring most recent changes to \code{master}.}
\label{fig:PullChanges1}
\end{figure}

Create the local topic branch. Use a meaningful name for \textit{topic} (see
Section~\ref{par:NamingTopics} on page~\pageref{par:NamingTopics}).

\begin{minted}[baselinestretch=1,fontsize=\footnotesize,linenos=false,bgcolor=ltgray]{bash}
git checkout -b topic
\end{minted}

\begin{figure}
\centering
\begin{tikzpicture}
\gitDAG[grow right sep = 1em, branch down = 3em]{
  -- C1 -- C2
};
\gittag
{master}       % node name and text
{above=of C2} % node placement
{C2}          % target
\gittag
{topic}       % node name and text
{below=of C2} % node placement
{C2}          % target
\end{tikzpicture}
\itkcaption{Create a local \code{topic} branch.}
\label{fig:CreateLocalBranch}
\end{figure}

This is where the real work happens. Edit, stage, and commit files
repeatedly as needed for your work.

During this step, avoid the \ref{par:UrgeToMerge} from an integration branch.
Keep your commits focused on the topic at hand.

\begin{minted}[baselinestretch=1,fontsize=\footnotesize,linenos=false,bgcolor=ltgray]{bash}
edit files
git add -- files
git commit
\end{minted}

\begin{figure}
\centering
\begin{tikzpicture}
\gitDAG[grow right sep = 1em, branch down = 3em]{
  C1 -- C2 -- {
  "" [placeholder];
    C3
  }
};
\gittag
{master}       % node name and text
{right=of C2} % node placement
{C2}          % target
\gittag
{topic}       % node name and text
{right=of C3} % node placement
{C3}          % target
\end{tikzpicture}
\itkcaption{Commit changes.}
\label{fig:CommitChanges}
\end{figure}

\begin{minted}[baselinestretch=1,fontsize=\footnotesize,linenos=false,bgcolor=ltgray]{bash}
edit files
git add -- files
git commit
\end{minted}

\begin{figure}
\centering
\begin{tikzpicture}
\gitDAG[grow right sep = 1em, branch down = 3em]{
  C1 -- C2 -- {
  "" [placeholder];
  C3 -- C4
}
};
\gittag
{master}       % node name and text
{right=of C2} % node placement
{C2}          % target
\gittag
{topic}       % node name and text
{right=of C4} % node placement
{C4}          % target
\end{tikzpicture}
\itkcaption{Commit last changes.}
\label{fig:CommitLastChanges}
\end{figure}

When the topic is ready for publication it must be merged into \textbf{master}.
It should be the current local branch when you merge.

Switch to \textbf{master} and update it.

\begin{minted}[baselinestretch=1,fontsize=\footnotesize,linenos=false,bgcolor=ltgray]{bash}
git checkout master
git pull
\end{minted}

\begin{figure}
\centering
\begin{tikzpicture}
\gitDAG[grow right sep = 1em, branch down = 3em]{
  C1 -- C2 -- {
  "" [placeholder];
  C3 -- C4
  }
  };
\gittag
{master}       % node name and text
{right=of C2} % node placement
{C2}          % target
\gittag
{topic}       % node name and text
{right=of C4} % node placement
{C4}          % target
\end{tikzpicture}
\itkcaption{Checkout \textit{master} and pull changes.}
\label{fig:SwitchToMasterAndPull}
\end{figure}

Merge the topic and test it.

\begin{minted}[baselinestretch=1,fontsize=\footnotesize,linenos=false,bgcolor=ltgray]{bash}
git merge topic
\end{minted}

\begin{figure}
\centering
\begin{tikzpicture}
\gitDAG[grow right sep = 1em, branch down = 3em]{
  C1 -- C2 -!- {
    "" [placeholder];
    C3 -- C4 -!- {
      "" [placeholder]; }
  } -!- C5;  {C2 -- C5}; {C2 -- C3}; {C4 -- C5};
};
\gittag
{master}       % node name and text
{right=of C5} % node placement
{C5}          % target
\gittag
{topic}       % node name and text
{right=of C4} % node placement
{C4}          % target
\end{tikzpicture}
\itkcaption{Merge \code{topic} branch into \code{master}.}
\label{fig:MergeTopicIntoMaster1}
\end{figure}

See Section~\ref{par:MultipleIntegrationBranches} on
page~\pageref{par:MultipleIntegrationBranches} to resolve any conflict
that may arise when merging.

Finally, publish the change.

\begin{minted}[baselinestretch=1,fontsize=\footnotesize,linenos=false,bgcolor=ltgray]{bash}
git push origin master
\end{minted}

\begin{figure}
\centering
\begin{tikzpicture}
\gitDAG[grow right sep = 1em, branch down = 3em]{
  C1 -- C2 -!- {
    "" [placeholder];
    C3 -- C4 -!- {
      "" [placeholder]; }
  } -!- C5;  {C2 -- C5}; {C2 -- C3}; {C4 -- C5};
};
\gittag
{origin/master}       % node name and text
{above=of C5} % node placement
{C5}          % target
\gittag
{master}       % node name and text
{below=of C5} % node placement
{C5}          % target
\gittag
{topic}       % node name and text
{right=of C4} % node placement
{C4}          % target
\end{tikzpicture}
\itkcaption{Publish the change.}
\label{fig:PushChanges1}
\end{figure}

See Section~\ref{subpar:RemoteEndHungUpUnexpectedly} on
page~\pageref{subpar:RemoteEndHungUpUnexpectedly} and
Section~\ref{subpar:NonFastForward} on page~\pageref{subpar:NonFastForward} to
resolve any conflict that may arise when publishing the change.


\paragraph{Mature Topic}
\label{par:MatureTopic}

When a topic is ready for inclusion in the next release, we merge it into
\textbf{master}.

Update \textbf{master} to get the latest work by others. We will merge the
\textit{topic} branch into it.

\begin{minted}[baselinestretch=1,fontsize=\footnotesize,linenos=false,bgcolor=ltgray]{bash}
git checkout master
\end{minted}

\begin{figure}
\centering
\begin{tikzpicture}
\gitDAG[grow right sep = 2em, branch down = 3em]{
  C1 -- C2 -- {
  "" [placeholder];
    C3 -- C4
    }
};
\gittag
{master}       % node name and text
{right=of C2} % node placement
{C2}          % target
\gittag
{topic}       % node name and text
{right=of C4} % node placement
{C4}          % target
\end{tikzpicture}
\itkcaption{Checkout \textit{master}.}
\label{fig:SwitchToMaster1}
\end{figure}

\begin{minted}[baselinestretch=1,fontsize=\footnotesize,linenos=false,bgcolor=ltgray]{bash}
git pull
\end{minted}

\begin{figure}
\centering
\begin{tikzpicture}
\gitDAG[grow right sep = 2em, branch down = 3em]{
  {C5 -- },
  C1 -- C2 -!- {
    "" [placeholder];
      C3 -- C4} -!- C6; {C2 -- C3}; {C2 -- C6}; {C5 -- C6}
};
\gittag
{master}       % node name and text
{right=of C6} % node placement
{C6}          % target
\gittag
{topic}       % node name and text
{right=of C4} % node placement
{C4}          % target
\end{tikzpicture}
\itkcaption{Pull latest changes into \textit{master}.}
\label{fig:PullChanges2}
\end{figure}

Merge the topic and test it.

\begin{minted}[baselinestretch=1,fontsize=\footnotesize,linenos=false,bgcolor=ltgray]{bash}
git merge topic
\end{minted}

\begin{figure}
\centering
\begin{tikzpicture}
\gitDAG[grow right sep = 2em, branch down = 3em]{
  {C5 -- },
  C1 -- C2 -!- {
    "" [placeholder];
    C3 -- C4} -!- C6 -- C7; {C2 -- C3}; {C2 -- C6}; {C5 -- C6}; {C4 -- C7}
};
\gittag
{master}       % node name and text
{right=of C7} % node placement
{C7}          % target
\gittag
{topic}       % node name and text
{right=of C4} % node placement
{C4}          % target
\end{tikzpicture}
\itkcaption{Merge the \code{topic} branch into master.}
\label{fig:MergeTopicIntoMaster2}
\end{figure}

See Section~\ref{par:MultipleIntegrationBranches} on
page~\pageref{par:MultipleIntegrationBranches} to resolve any conflict that may
arise when merging the branch.

Delete the local branch.

\begin{minted}[baselinestretch=1,fontsize=\footnotesize,linenos=false,bgcolor=ltgray]{bash}
git branch -d topic
\end{minted}

\begin{figure}
\centering
\begin{tikzpicture}
\gitDAG[grow right sep = 2em, branch down = 3em]{
  {C5 -- },
  C1 -- C2 -!- {
    "" [placeholder];
    C3 -- C4} -!- C6 -- C7; {C2 -- C3}; {C2 -- C6}; {C5 -- C6}; {C4 -- C7}
};
\gittag
{master}       % node name and text
{right=of C7} % node placement
{C7}          % target
\end{tikzpicture}
\itkcaption{Delete the local \code{topic} branch.}
\label{fig:DeleteLocalBranch1}
\end{figure}

Finally, publish the change.

\begin{minted}[baselinestretch=1,fontsize=\footnotesize,linenos=false,bgcolor=ltgray]{bash}
git push origin master
\end{minted}

\begin{figure}
\centering
\begin{tikzpicture}
\gitDAG[grow right sep = 2em, branch down = 3em]{
  {C5 -- },
  C1 -- C2 -!- {
    "" [placeholder];
    C3 -- C4} -!- C6 -- C7; {C2 -- C3}; {C2 -- C6}; {C5 -- C6}; {C4 -- C7}
};
\gittag
{origin/master}       % node name and text
{above=of C7} % node placement
{C7}          % target
\gittag
{master}       % node name and text
{below=of C7} % node placement
{C7}          % target
\end{tikzpicture}
\itkcaption{Publish the change.}
\label{fig:PushChanges2}
\end{figure}

See Section~\ref{subpar:RemoteEndHungUpUnexpectedly} on
page~\pageref{subpar:RemoteEndHungUpUnexpectedly} and
Section~\ref{subpar:NonFastForward} on page~\pageref{subpar:NonFastForward} to
resolve any conflict that may arise when publishing the change.


\paragraph{Old Topic}
\label{par:OldTopic}

Sometimes we need to continue work on an old topic that has already been merged
to an integration branch and for which we no longer have a local topic branch.
To revive an old topic, we create a local branch based on the last commit from
the topic (this is not one of the merges into an integration branch).

First we need to identify the commit by its hash. It is an ancestor of the
integration branch into which it was once merged, say \textbf{master}. Run
\code{git log} with the \code{--first-parent} option to view the integration
history:

\begin{minted}[baselinestretch=1,fontsize=\footnotesize,linenos=false,bgcolor=ltgray]{bash}
git log --first-parent master
commit 9057863...
Merge: 2948732 a348901
...
    Merge branch topicA

commit 2948732...
Merge: 1094687 b235725
...
    Merge branch topicB

commit 1094687...
Merge: 8267263 c715789
...
    Merge branch topicC
\end{minted}

Locate the merge commit for the topic of interest, say \textit{topicB}. Its
second parent is the commit from which we will restart work (\textit{b235725} in
this example).

Create a local topic branch starting from the commit identified above.

\begin{minted}[baselinestretch=1,fontsize=\footnotesize,linenos=false,bgcolor=ltgray]{bash}
git checkout -b topic b235725
\end{minted}

\begin{figure}
\centering
\begin{tikzpicture}
\gitDAG[grow right sep = 2em, branch down = 3em]{
  {C5 -- },
  C1 -- {C2 -- C6,
    C3 -- C4} -- C7; {C5 -- C6}
};
\gittag
{master}       % node name and text
{right=of C7} % node placement
{C7}          % target
\gittag
{topic b235725}       % node name and text
{right=of C4} % node placement
{C4}          % target
\end{tikzpicture}
\itkcaption{Local \textit{topic} branch started from a given commit and switch to it.}
\label{fig:CreateLocalBranchFromCommitAndSwitchToIt}
\end{figure}

Continue development on the topic.

\begin{minted}[baselinestretch=1,fontsize=\footnotesize,linenos=false,bgcolor=ltgray]{bash}
edit files
git add -- files
git commit
\end{minted}

\begin{figure}
\centering
\begin{tikzpicture}
\gitDAG[grow right sep = 2em, branch down = 3em]{
  {C5 -- },
  C1 -- {C2 -- C6 -- C7},
    C3 -- C4 -- C8; {C5 -- C6}; {C4 -- C7}
};
\gittag
{master}       % node name and text
{right=of C7} % node placement
{C7}          % target
\gittag
{topic}       % node name and text
{right=of C8} % node placement
{C8}          % target
\end{tikzpicture}
\itkcaption{Commit files to \textit{topic}.}
\label{fig:CommitFilesToTopic1}
\end{figure}

\begin{minted}[baselinestretch=1,fontsize=\footnotesize,linenos=false,bgcolor=ltgray]{bash}
edit files
git add -- files
git commit
\end{minted}

\begin{figure}
\centering
\begin{tikzpicture}
\gitDAG[grow right sep = 2em, branch down = 3em]{
  {C5 -- },
  C1 -- {C2 -- C6 -- C7,
    C3 -- C4 -- C8 -- C9}; {C5 -- C6}; {C4 -- C7}
};
\gittag
{master}       % node name and text
{right=of C7} % node placement
{C7}          % target
\gittag
{topic}       % node name and text
{right=of C9} % node placement
{C9}          % target
\end{tikzpicture}
\itkcaption{Further continue developing and commit files to \textit{topic}.}
\label{fig:CommitFilesToTopic2}
\end{figure}

When the new portion of the topic is ready, merge it into \textbf{master} and test.

\begin{minted}[baselinestretch=1,fontsize=\footnotesize,linenos=false,bgcolor=ltgray]{bash}
git checkout master
git pull
git merge topic
\end{minted}

\begin{figure}
\centering
\begin{tikzpicture}
\gitDAG[grow right sep = 2em, branch down = 3em]{
  {C5 -- },
  C1 -- {C2 -- C6 -- C7 -- C10,
    C3 -- C4 -- C8 -- C9}; {C5 -- C6}; {C4 -- C7}; {C9 -- C10}
};
\gittag
{master}       % node name and text
{right=of C10} % node placement
{C10}          % target
\gittag
{topic}       % node name and text
{right=of C9} % node placement
{C9}          % target
\end{tikzpicture}
\itkcaption{Merge into \textit{master}.}
\label{fig:MergeTopicIntoMaster3}
\end{figure}

Publish \textbf{master}.

\begin{minted}[baselinestretch=1,fontsize=\footnotesize,linenos=false,bgcolor=ltgray]{bash}
git push origin master
\end{minted}


\paragraph{Dependent Topic}
\label{par:DependentTopic}

Occasionally you may realize that you need the work from another topic to
complete work on your topic. In this case your topic depends on the other
topic, so merging the other topics into yours is legitimate (see
Section~\ref{par:LegitimateMerges} on page~\pageref{par:LegitimateMerges}). Do
not merge an integration branch that has the \textit{other-topic} branch name.
Use the instructions below to merge only the \textit{other-topic} branch
without getting everything else.

Fetch the upstream integration branch that has the \textit{other-topic} branch,
say \textbf{master}.

\begin{minted}[baselinestretch=1,fontsize=\footnotesize,linenos=false,bgcolor=ltgray]{bash}
git fetch origin
\end{minted}

\begin{figure}
\centering
\begin{tikzpicture}
\gitDAG[grow right sep = 2em, branch down = 3em]{
   {C5 -!- C6 -- },
   C1 -!- {C2 -- C7 -- C8,
     C3 -- C4,
     C9 -- C10}; {C1 -- C2}; {C1 -- C9}; {C2 -- C3};
    {C6 -- C7}; {C4 -- C8}; {C5 -- C2};
};
\gittag
{extra-topic}       % node name and text
{right=of C6} % node placement
{C6}          % target
\gittag
{origin/master}       % node name and text
{right=of C8} % node placement
{C8}          % target
\gittag
{0a398e5}       % node name and text
{right=of C4} % node placement
{C4}          % target
\gittag
{topic}       % node name and text
{right=of C10} % node placement
{C10}          % target
\end{tikzpicture}
\itkcaption{Fetch the upstream integration branch \textit{other-topic}.}
\label{fig:FetchUpstreamBranch}
\end{figure}

Use \code{git log --first-parent origin/master} to find the commit that merges
\textit{other-topic}. The commit message gives you the name of the other topic
branch (we use \textit{other-topic} here as a placeholder). The second parent of
the commit (\code{0a398e5} in this example) is the end of the
\textit{other-topic} branch. Create a local branch from that commit.

\begin{minted}[baselinestretch=1,fontsize=\footnotesize,linenos=false,bgcolor=ltgray]{bash}
git branch other-topic 0a398e5
\end{minted}

\begin{figure}
\centering
\begin{tikzpicture}
\gitDAG[grow right sep = 2em, branch down = 3em]{
   {C5 -!- C6 -- },
   C1 -!- {C2 -- C7 -- C8,
     C3 -- C4,
     C9 -- C10}; {C1 -- C2}; {C1 -- C9}; {C2 -- C3};
    {C6 -- C7}; {C4 -- C8}; {C5 -- C2};
};
\gittag
{extra-topic}       % node name and text
{right=of C6} % node placement
{C6}          % target
\gittag
{origin/master}       % node name and text
{right=of C8} % node placement
{C8}          % target
\gittag
{other-topic}       % node name and text
{right=of C4} % node placement
{C4}          % target
\gittag
{topic}       % node name and text
{right=of C10} % node placement
{C10}          % target
\end{tikzpicture}
\itkcaption{Create a local branch \textit{other-topic} from commit \textit{0a398e5}.}
\label{fig:CreateLocalBranchFromCommit}
\end{figure}

Merge the \textit{other-topic} branch into your topic.

\begin{minted}[baselinestretch=1,fontsize=\footnotesize,linenos=false,bgcolor=ltgray]{bash}
git merge other-topic
git branch -d other-topic
\end{minted}

\begin{figure}
\centering
\begin{tikzpicture}
\gitDAG[grow right sep = 2em, branch down = 3em]{
   {C5 -!- C6 -- },
   C1 -!- {C2 -- C7 -- C8,
     C3 -- C4,
     C9 -- C10 -- C11}; {C1 -- C2}; {C1 -- C9}; {C2 -- C3};
    {C6 -- C7}; {C4 -- C8}; {C5 -- C2}; {C4 -- C11}
};
\gittag
{extra-topic}       % node name and text
{right=of C6} % node placement
{C6}          % target
\gittag
{origin/master}       % node name and text
{right=of C8} % node placement
{C8}          % target
\gittag
{topic}       % node name and text
{right=of C11} % node placement
{C11}          % target
\end{tikzpicture}
\itkcaption{Merge \textbf{other-branch} into \textit{topic} branch.}
\label{fig:MergeBranch}
\end{figure}

(It is also possible to run \code{git merge 0a398e5} and then use
\code{git commit --amend} to write a nice commit message.)

The \textit{topic} branch now looks like this:

\begin{figure}
\centering
\begin{tikzpicture}
\gitDAG[grow right sep = 2em, branch down = 3em]{
  {C5 -!- },
  C1 -!- {C2,
    C3 -- C4,
    C9 -- C10 -- C11}; {C1 -- C2}; {C1 -- C9}; {C2 -- C3}; {C4 -- C11}
};
\gittag
{topic}       % node name and text
{right=of C11} % node placement
{C11}          % target
\end{tikzpicture}
\itkcaption{\textit{topic} branch shape.}
\label{fig:BranchShape}
\end{figure}

Note that after the merge, the \textit{other-topic} is reachable from your topic
but the extra-topic has not been included. By not merging from the integration
branch we avoided bringing in an unnecessary dependency on the
\textit{extra-topic}. Furthermore, the message ``Merge branch 'other-topic' into
topic'' is very informative about the purpose of the merge. Merging the whole
integration branch would not be so clear.


\paragraph{Merge Integration Branches}
\label{par:MergeIntegrationBranches}

Each published integration branch (see Section~\ref{subsubsec:PublishedBranches}
on page~\pageref{subsubsec:PublishedBranches}) has a defined level of
stability. Express this relationship by merging more-stable branches into
less-stable branches to ensure that they do not diverge. After merging a mature
topic to \textbf{master}, we merge \textbf{master} into \textbf{release}:

Update \textbf{master} and then \textbf{release}:

\begin{minted}[baselinestretch=1,fontsize=\footnotesize,linenos=false,bgcolor=ltgray]{bash}
git checkout master
git pull
git checkout release
git pull
\end{minted}

\begin{figure}
\centering
\begin{tikzpicture}
\gitDAG[grow right sep = 2em, branch down = 3em]{
  {C5 -- },
    C1 -!- {C2 -- C6 -- C7,
    C3 -- C4,
    C8 -- C9}; {C1 -- C2}; {C2 -- C3}; {C1 -- C2}; {C4 -- C7}; {C4 -- C9}; {C5 -- C6}
};
\gittag
{master}       % node name and text
{right=of C7} % node placement
{C7}          % target
\gittag
{release}       % node name and text
{right=of C9} % node placement
{C9}          % target
\end{tikzpicture}
\itkcaption{Update \textit{master} and \textit{release}.}
\label{fig:UpdateBranches}
\end{figure}

Merge \textbf{master} into \textbf{release}:

\begin{minted}[baselinestretch=1,fontsize=\footnotesize,linenos=false,bgcolor=ltgray]{bash}
git merge master
\end{minted}

\begin{figure}
\centering
\begin{tikzpicture}
\gitDAG[grow right sep = 2em, branch down = 3em]{
  {C5 -- },
  C1 -!- {C2 -- C6 -- C7,
  C3 -- C4,
  C8 -- C9 -- C10}; {C1 -- C2}; {C2 -- C3}; {C1 -- C2}; {C4 -- C7};
  {C4 -- C9}; {C5 -- C6}; {C7 -- C10}
  };
\gittag
{master}       % node name and text
{right=of C7} % node placement
{C7}          % target
\gittag
{release}       % node name and text
{right=of C10} % node placement
{C10}          % target
\end{tikzpicture}
\itkcaption{Merge \textit{master} into \textit{release}.}
\label{fig:MergeMasterIntoRelease}
\end{figure}

Finally, publish the change.

\begin{minted}[baselinestretch=1,fontsize=\footnotesize,linenos=false,bgcolor=ltgray]{bash}
git push origin release
\end{minted}

\begin{figure}
\centering
\begin{tikzpicture}
\gitDAG[grow right sep = 2em, branch down = 3em]{
  {C5 -- },
    C1 -!- {C2 -- C6 -- C7,
    C3 -- C4,
    C8 -- C9 -- C10}; {C1 -- C2}; {C2 -- C3}; {C1 -- C2}; {C4 -- C7};
    {C4 -- C9}; {C5 -- C6}; {C7 -- C10}
};
\gittag
{master}       % node name and text
{right=of C7} % node placement
{C7}          % target
\gittag
{release}       % node name and text
{right=of C10} % node placement
{C10}          % target
\gittag
{origin/release}       % node name and text
{below=of C10} % node placement
{C10}          % target
\end{tikzpicture}
\itkcaption{Publish to \textit{release}.}
\label{fig:PushChangesToRelease1}
\end{figure}

See Section~\ref{subpar:RemoteEndHungUpUnexpectedly} on
page~\pageref{subpar:RemoteEndHungUpUnexpectedly} and
Section~\ref{subpar:NonFastForward} on page~\pageref{subpar:NonFastForward} to
resolve any conflict that may arise when publishing the change.

\subsubsection{Discussion}
\label{subpar:Discussion}

\paragraph{History Shape}
\label{par:HistoryShape}

The history graphs produced by this workflow may look complex compared to the
fully linear history produced by a rebase workflow (used by CVS and Subversion):

\begin{figure}
\centering
\begin{tikzpicture}
\gitDAG[grow right sep = 2em, branch down = 3em]{
  {C5 -- },
  C1 -!- {C2 -- C6 -- C7 -- C15,
    C3 -- C4 -- C8 -- C14},
   {C9 -- C11 -- C13 -- C16},
   {C10 -!- C12}; {C1 -- C2}; {C2 -- C3};
   {C5 -- C6}; {C4 -- C7}; {C4 -- C11}; {C14 -- C15}; {C14 -- C16}; {C10 -- C9}; {C12 -- C13};
};
\gittag
{master}       % node name and text
{right=of C15} % node placement
{C15}          % target
\gittag
{topic}       % node name and text
{right=of C14} % node placement
{C14}          % target
\gittag
{release}       % node name and text
{right=of C16} % node placement
{C16}          % target
\end{tikzpicture}
\itkcaption{Complex history graph in Git.}
\label{fig:ComplexHistory}
\end{figure}

However, consider the shape of history along each branch. We can view it using
Git's \textit{--first-parent} option. It traverses history by following only the
first parent of each merge. The first parent is the commit that was currently
checked out when the git merge command was invoked to create the merge commit.
By following only the first parent, we see commits that logically belong to a
specific branch.

\begin{minted}[baselinestretch=1,fontsize=\footnotesize,linenos=false,bgcolor=ltgray]{bash}
git log --first-parent topic
\end{minted}

\begin{figure}
\centering
\begin{tikzpicture}
\gitDAG[grow right sep = 1em, branch down = 3em]{
  C2 -- {
    "" [placeholder];
    C3 -- C4 -- C8 -- C14
  }
};
\gittag
{topic}       % node name and text
{right=of C14} % node placement
{C14}          % target
\end{tikzpicture}
\itkcaption{Parent commit in \textit{topic} branch.}
\label{fig:ParentCommitInTopicBranch}
\end{figure}


\begin{minted}[baselinestretch=1,fontsize=\footnotesize,linenos=false,bgcolor=ltgray]{bash}
git log --first-parent master
\end{minted}

\begin{figure}
\centering
\begin{tikzpicture}
\gitDAG[grow right sep = 1em, branch down = 3em]{
  C5, {
    C1 -- C2 -- C6 -- C7 -- C15
  }, {C4 -!- C14}; {C4 -- C7}; {C5 -- C6}; {C14 -- C15}
};
\gittag
{master}       % node name and text
{right=of C15} % node placement
{C15}          % target
\end{tikzpicture}
\itkcaption{Parent commit in \textit{master} branch.}
\label{fig:ParentCommitInMasterBranch}
\end{figure}

\begin{minted}[baselinestretch=1,fontsize=\footnotesize,linenos=false,bgcolor=ltgray]{bash}
git log --first-parent release
\end{minted}


\begin{figure}
\centering
\begin{tikzpicture}
\gitDAG[grow right sep = 1em, branch down = 3em]{
  C4 -!- C14 -!- {
    "" [placeholder];
    C9 -- C11 -- C13 -- C16
  }, {C10 -!- C12}; {C4 -- C11}; {C10 -- C9}; {C12 -- C13}; {C14 -- C16}
};
\gittag
{release}       % node name and text
{right=of C16} % node placement
{C16}          % target
\end{tikzpicture}
\itkcaption{Parent commit in \textit{release} branch.}
\label{fig:ParentCommitInReleaseBranch}
\end{figure}

Each branch by itself looks linear and has only commits with a specific purpose.
The history behind each commit is unique to that purpose. Topic branches are
independent, containing only commits for their specific feature or fix.
Integration branches consist of merge commits that integrate topics together.

Note that achieving the nice separation of branches requires understanding of
the above development procedure and strict adherence to it.


\paragraph{Naming Topics}
\label{par:NamingTopics}

This section uses the placeholder \textit{topic} in place of a real topic name.
In practice, substitute for a meaningful name. Name topics like you might name
functions: concise but precise. A reader should have a general idea of the
feature or fix to be developed given just the branch name.

Note that topic names are not published as branch heads on the server, so no one
will ever see a branch by your topic name unless they create it themselves.
However, the names do appear in the default merge commit message:

\begin{minted}[baselinestretch=1,fontsize=\footnotesize,linenos=false,bgcolor=ltgray]{bash}
git checkout master
git merge topic
git show
...
   Merge branch 'topic' into master
...
\end{minted}

These merge commits appear on the integration branches and should therefore
describe the changes they integrate. Running \code{git log --first-parent} as
described in Section~\ref{par:HistoryShape} will show only these merge commits,
so their messages should be descriptive of the changes made on their topics. If
you did not choose a good branch name, or feel that the merge needs more
explanation than the branch name provides, amend the commit to update the
message by hand:

\begin{minted}[baselinestretch=1,fontsize=\footnotesize,linenos=false,bgcolor=ltgray]{bash}
git commit --amend
Merge branch 'topic' into master
(edit the message)
\end{minted}


\paragraph{Urge to Merge}
\label{par:UrgeToMerge}

Avoid the ``urge to merge'' from an integration branch into your topic. Keep
commits on your topic focused on the feature or fix under development.

\paragraph{Habitual Merges}
\label{par:HabitualMerges}

Merge your work with others when you are finished with it by merging into an
integration branch as documented above. Avoid habitual merges from an
integration branch; doing so introduces unnecessary dependencies and complicates
the shape of history (see Section~\ref{par:HistoryShape} on
page~\pageref{par:HistoryShape}).

Many developers coming from centralized version control systems have trained
themselves to regularly update their work tree from the central repository (e.g.
``cvs update''). With those version control systems this was a good habit
because they did not allow you to commit without first integrating your work
with the latest from the server. When integrating the local and remote changes
resulted in conflicts, developers were forced to resolve the conflicts before
they could commit. A mistake during conflict resolution could result in loss of
work because the local changes might have been lost. By regularly updating from
the server, developers hoped to avoid this loss of work by resolving conflicts
incrementally.

Developers using Git do not face this problem. Instead, one should follow a
simple motto: ``commit first, integrate later''. There is no risk that your work
will be lost during conflict resolution because all your changes have been
safely committed before attempting to merge. If you make a mistake while merging,
you always have the option to throw away the merge attempt and start over with a
clean tree.


\paragraph{Legitimate Merges}
\label{par:LegitimateMerges}

One reason to merge other work into your topic is when you realize that your
topic depends on it. See Section~\ref{par:DependentTopic} on
page~\pageref{par:DependentTopic} for help with this case.

Occasionally one may merge directly from \textbf{master} if there is a good
reason. This is rare, so bring up the reason on your project discussion forum
first. Never merge \textbf{release} into a topic under any circumstances!!!


\subsubsection{Troubleshooting}
\label{subsubsec:Troubleshooting}

Here we document problems one might encounter while following the workflow
instructions above. This is not a general Git troubleshooting page.

\paragraph{Trouble Merging}
\label{par:TroubleMerging}

%TODO: Write this sub-section and link sub-sub-sections from git merge commands above.

\paragraph{Trouble Pushing}
\label{par:TroublePushing}

\subparagraph{Remote End Hung up Unexpectedly}
\label{subpar:RemoteEndHungUpUnexpectedly}

Pushing may fail with this error:

\begin{minted}[baselinestretch=1,fontsize=\footnotesize,linenos=false,bgcolor=ltgray]{bash}
git push
fatal: The remote end hung up unexpectedly
\end{minted}

This likely means that you have set a \textit{push URL} for the remote
repository. You can see the URL to which it tries to push using \code{-v}:

\begin{minted}[baselinestretch=1,fontsize=\footnotesize,linenos=false,bgcolor=ltgray]{bash}
git push -v
Pushing to git://public.kitware.com/Project.git
fatal: The remote end hung up unexpectedly
\end{minted}

The \code{git://} repository URL may not be used for pushing; it is meant for
efficient read-only anonymous access only. Instead you need to configure a
SSH-protocol URL for pushing:

\begin{minted}[baselinestretch=1,fontsize=\footnotesize,linenos=false,bgcolor=ltgray]{bash}
git config remote.origin.pushurl git@public.kitware.com:Project.git
\end{minted}

(Note that \code{pushurl} requires Git >= 1.6.4. Use just \code{url} for Git <
1.6.4.). The URL in the above example is a placeholder. In practice, use th
push URL documented for your repository.

The above assumes that you want to push to the same repository that you
originally cloned. To push elsewhere, see help for
\href{https://git-scm.com/docs/git-push}{git push} and
\href{https://git-scm.com/docs/git-remote}{git remote}.


\subparagraph{Non-Fast-Forward}
\label{subpar:NonFastForward}

When trying to publish new merge commits on an integration branch, perhaps
\textbf{release}, the final push may fail:

\begin{minted}[baselinestretch=1,fontsize=\footnotesize,linenos=false,bgcolor=ltgray]{bash}
git push origin release
To ...
! [rejected]        release -> release (non-fast-forward)
error: failed to push some refs to '...'
To prevent you from losing history, non-fast-forward updates were rejected
Merge the remote changes before pushing again.  See the 'Note about
fast-forwards' section of 'git push --help' for details.
\end{minted}

This means that the server's \textbf{release} refers to a commit that is not
reachable from the \textbf{release} you are trying to push:

\begin{figure}
\centering
\begin{tikzpicture}
\gitDAG[grow right sep = 2em, branch down = 3em]{
  C1 -- C2 -- {
    "" [placeholder];
    C3 -- C4},
  C5;
  -- C6 -- C8,
  C7; {C4 -- C5}; {C7 -- C8}
};
\gittag
{master}       % node name and text
{right=of C2} % node placement
{C2}          % target
\gittag
{topic}       % node name and text
{right=of C4} % node placement
{C4}          % target
\gittag
{release}       % node name and text
{right=of C5} % node placement
{C5}          % target
\gittag
{origin/release}       % node name and text
{right=of C8} % node placement
{C8}          % target
\gittag
{other-topic}       % node name and text
{right=of C7} % node placement
{C7}          % target
\end{tikzpicture}
\itkcaption{Unreachable \textit{origin/release} branch.}
\label{fig:UnreachableBranch}
\end{figure}

This is the Git equivalent to when \textit{cvs commit} complains that your file
is not up-to-date, but now it applies to the whole project and not just one
file. Git is telling you that it cannot update \textbf{release} on the server to
point at your merge commit because that would throw away someone else's work
(such as \textit{other-topic}). There are a few possible causes, all of which
mean you have not yet integrated your work with the latest from
\textit{upstream}:
\begin{itemize}
\item You forgot to run \code{git pull} before \code{git merge} so you did not
have everything from \textit{upstream}.
\item Someone else managed to merge and push something into \textbf{release}
since you last ran \code{git pull}.
\end{itemize}

Some Git guides may tell you to just \code{git pull} again to merge
\textit{upstream} work into yours. That approach is not compatible with the goals
of this workflow. We want to preserve a clean shape of history (see
Section~\ref{par:HistoryShape} on page~\pageref{par:HistoryShape}).

The solution is to throw away your previous merge and try again, but this time
start from the latest \textit{upstream} work:

\begin{minted}[baselinestretch=1,fontsize=\footnotesize,linenos=false,bgcolor=ltgray]{bash}
git reset --hard origin/release
\end{minted}

\begin{figure}
\centering
\begin{tikzpicture}
\gitDAG[grow right sep = 2em, branch down = 3em]{
  C1 -- C2 -- {
    "" [placeholder];
    C3 -- C4},
  -- C6 -- C8,
  C7; {C7 -- C8}
};
\gittag
{master}       % node name and text
{right=of C2} % node placement
{C2}          % target
\gittag
{topic}       % node name and text
{right=of C4} % node placement
{C4}          % target
\gittag
{origin/release}       % node name and text
{right=of C8} % node placement
{C8}          % target
\gittag
{release}       % node name and text
{below=of C8} % node placement
{C8}          % target
\gittag
{other-topic}       % node name and text
{below=of C7} % node placement
{C7}          % target
\end{tikzpicture}
\itkcaption{Start from latest upstream commit.}
\label{fig:StartFromLatestUpstreamCommit}
\end{figure}

\begin{minted}[baselinestretch=1,fontsize=\footnotesize,linenos=false,bgcolor=ltgray]{bash}
git merge topic
\end{minted}

\begin{figure}
\centering
\begin{tikzpicture}
\gitDAG[grow right sep = 2em, branch down = 3em]{
  C1 -- C2 -- {
    "" [placeholder];
    C3 -- C4},
  -- C6 -- C8 -- C9,
  C7; {C7 -- C8}; {C4 -- C9}
};
\gittag
{master}       % node name and text
{right=of C2} % node placement
{C2}          % target
\gittag
{topic}       % node name and text
{right=of C4} % node placement
{C4}          % target
\gittag
{release}       % node name and text
{right=of C9} % node placement
{C9}          % target
\gittag
{other-topic}       % node name and text
{right=of C7} % node placement
{C7}          % target
\end{tikzpicture}
\itkcaption{Merge the \textit{topic} branch into the local \textit{release} branch.}
\label{fig:MergeTopicIntoRelease}
\end{figure}

Now your \textbf{release} can reach the upstream work as well as yours. Publish
it.

\begin{minted}[baselinestretch=1,fontsize=\footnotesize,linenos=false,bgcolor=ltgray]{bash}
git push origin release
\end{minted}

\begin{figure}
\centering
\begin{tikzpicture}
\gitDAG[grow right sep = 2em, branch down = 3em]{
  C1 -- C2 -- {
    "" [placeholder];
    C3 -- C4},
  -- C6 -- C8 -- C9,
  C7; {C7 -- C8}; {C4 -- C9}
};
\gittag
{master}       % node name and text
{right=of C2} % node placement
{C2}          % target
\gittag
{topic}       % node name and text
{right=of C4} % node placement
{C4}          % target
\gittag
{origin/release}       % node name and text
{right=of C9} % node placement
{C9}          % target
\gittag
{release}       % node name and text
{below=of C9} % node placement
{C9}          % target
\gittag
{other-topic}       % node name and text
{below=of C7} % node placement
{C7}          % target
\end{tikzpicture}
\itkcaption{Publish the \textit{release} branch.}
\label{fig:PushChangesToRelease}
\end{figure}

See \href{https://git-scm.com/docs/git-rerere}{git rerere} to help avoid
resolving the same conflicts on each merge attempt.


\subparagraph{First-Parent Sequence Not Preserved}
\label{subpar:FirstParentSequenceNotPreserved}

One goal of this workflow is to preserve a clean shape of history (see
Section~\ref{par:HistoryShape} on page~\pageref{par:HistoryShape}). This means
that a \code{--first-parent} traversal of an integration branch, such as
\textbf{master}, should see only the merge commits that integrate topics into
the branch:

\begin{figure}
\centering
\begin{tikzpicture}
\gitDAG[grow right sep = 2em, branch down = 3em]{
  {C1 -- } -!- {C5 -- },
  {[nodes=placeholder commits] "..."} -- C2 -- C4 -- C6 -- C8,
  {C3 -- } -!- {C7 -- };
  {C1 -- C2}; {C3 -- C4}; {C5 -- C6}; {C7 -- C8}
};
\gittag
{master}       % node name and text
{right=of C8} % node placement
{C8}          % target
\end{tikzpicture}
\itkcaption{Traversal of the \textit{master} integration branch.}
\label{fig:TraversalOfMasterBranch}
\end{figure}

The commits on the individual topic branches are not included in the traversal.
This provides a medium-level overview of the development of the project.

We enforce the shape of history on the server's integration branches using an
update hook at push-time. Each update must point its branch at a new commit from
which a \code{first-parent} traversal reaches the old head of the branch:

\begin{figure}
\centering
\begin{tikzpicture}
\gitDAG[grow right sep = 2em, branch down = 3em]{
  {F -- } -!- {G -- },
  {[nodes=placeholder commits] "..."} -- E -- D -- C -- B -- A -- M,
  {H -- D} -!- {J -- B} -!- {{B -- U -- T -- M}};
  {F -- E}; {G -- C}
  };
\gittag
{master}       % node name and text
{right=of M} % node placement
{M}          % target
\gittag
{master@{1}}       % node name and text
{above=of A} % node placement
{A}          % target
\gittag
{topic}       % node name and text
{right=of T} % node placement
{T}          % target
\end{tikzpicture}
\itkcaption{Server's integration branch history shape.}
\label{fig:IntegrationBranchHistory}
\end{figure}

A first-parent traversal of master from \textit{before} the update
(\code{master@{1}}) sees \code{A} \code{B} \code{C} \code{D}:

\begin{figure}
\centering
\begin{tikzpicture}
\gitDAG[grow right sep = 2em, branch down = 3em]{
  {F -- } -!- {G -- } -!- {K -- },
  {[nodes=placeholder commits] "..."} -- E -- D -- C -- B -- A,
  {H -- D} -!- {J -- B} -!- {F -- E}; {G -- C}; {K -- A}
};
\gittag
{master@{1}}       % node name and text
{right=of A} % node placement
{A}          % target
\end{tikzpicture}
\itkcaption{First parent traversal of \textit{master} before update.}
\label{fig:FirstParentTraversalBeforeUpdate}
\end{figure}

A first-parent traversal of \textbf{master} from \textit{after} the update sees
\code{M} \code{A} \code{B} \code{C} \code{D}:

\begin{figure}
\centering
\begin{tikzpicture}
\gitDAG[grow right sep = 2em, branch down = 3em]{
  {F -- } -!- {G -- } -!- {K -- },
  {[nodes=placeholder commits] "..."} -- E -- D -- C -- B -- A -- M,
  {H -- D} -!- {J -- B} -!- {T -- M}; {F -- E}; {G -- C}; {K -- A}
};
\gittag
{master}       % node name and text
{right=of M} % node placement
{M}          % target
\end{tikzpicture}
\itkcaption{First parent traversal of \textit{master} after update.}
\label{fig:FirstParentTraversalAfterUpdate}
\end{figure}

The above assumes correct history shape. Now, consider what happens if merge
\code{M} is incorrectly made on the topic branch:

\begin{figure}
\centering
\begin{tikzpicture}
\gitDAG[grow right sep = 2em, branch down = 3em]{
  {F -- } -!- {G -- } -!- {K -- },
  {[nodes=placeholder commits] "..."} -- E -- D -- C -- B -- A,
  {H -- D} -!- {J -- B} -!- {U -- T -- M'}; {F -- E}; {G -- C}; {K -- A}; {U -- B}; {M' -- A};
};
\gittag
{master@{1}}       % node name and text
{above=of A} % node placement
{A}          % target
\gittag
{master}       % node name and text
{right=of M'} % node placement
{M'}          % target
\gittag
{topic}       % node name and text
{below=of T} % node placement
{T}          % target
\end{tikzpicture}
\itkcaption{Incorrect merge of \code{M} on branch \textit{topic}.}
\label{fig:IncorrectMerge}
\end{figure}

Now a first-parent traversal of \textbf{master} from after the update sees
\code{M'} \code{T} \code{U} \code{B} \code{C} \code{D}:

\begin{figure}
\centering
\begin{tikzpicture}
\gitDAG[grow right sep = 2em, branch down = 3em]{
  {F -- } -!- {G -- },
  {[nodes=placeholder commits] "..."} -- E -- D -- C -- B,
  {H -- D} -!- {J -- B} -!- {U -- T -- M'}; {F -- E}; {G -- C}; {U -- B}; {M' -- };
};
\gittag
{master}       % node name and text
{right=of M'} % node placement
{M'}          % target
\gittag
{topic}       % node name and text
{below=of T} % node placement
{T}          % target
\end{tikzpicture}
\itkcaption{First-parent traversal of \textit{master} branch.}
\label{fig:FirstParentTraversal}
\end{figure}

This not only shows details of the topic branch, but skips over \code{A}
altogether! Our update hooks will reject the push in this case because the
new \textbf{master} cannot see the old one in a first-parent traversal.

There are a few possible causes and solutions to the above problem, but all
involve non-strict compliance with the workflow instructions. A likely cause is
that you did not create a local \textit{topic} branch but instead committed
directly on \textbf{master} and then pulled from \textit{upstream} before
pushing:

\begin{minted}[baselinestretch=1,fontsize=\footnotesize,linenos=false,bgcolor=ltgray]{bash}
wrong$ git checkout master
\end{minted}

\begin{figure}
\centering
\begin{tikzpicture}
\gitDAG[grow right sep = 2em, branch down = 3em]{
  {F -- } -!- {G -- },
  {[nodes=placeholder commits] "..."} -- E -- D -- C -- B ,
  {H -- D} -!- {J -- B}; {F -- E}; {G -- C}
};
\gittag
{origin/master}       % node name and text
{above=of B} % node placement
{B}          % target
\gittag
{master}       % node name and text
{below=of B} % node placement
{B}          % target
\end{tikzpicture}
\itkcaption{Checkout \textit{master}.}
\label{fig:SwitchToMaster2}
\end{figure}

\begin{minted}[baselinestretch=1,fontsize=\footnotesize,linenos=false,bgcolor=ltgray]{bash}
wrong$ edit files
wrong$ git add files
wrong$ git commit
\end{minted}

\begin{figure}
\centering
\begin{tikzpicture}
\gitDAG[grow right sep = 2em, branch down = 3em]{
  {F -- } -!- {G -- },
  {[nodes=placeholder commits] "..."} -- E -- D -- C -- B,
  {H -- D} -!- {J -- B} -!- {{B -- U}}; {F -- E}; {G -- C}
};
\gittag
{origin/master}       % node name and text
{above=of B} % node placement
{B}          % target
\gittag
{master}       % node name and text
{right=of U} % node placement
{U}          % target
\end{tikzpicture}
\itkcaption{Commit on \textit{master} branch.}
\label{fig:WrongCommitOnMasterBranch}
\end{figure}

\begin{minted}[baselinestretch=1,fontsize=\footnotesize,linenos=false,bgcolor=ltgray]{bash}
wrong$ edit files
wrong$ git add files
wrong$ git commit
\end{minted}

\begin{figure}
\centering
\begin{tikzpicture}
\gitDAG[grow right sep = 2em, branch down = 3em]{
  {F -- } -!- {G -- },
  {[nodes=placeholder commits] "..."} -- E -- D -- C -- B -- A -- M,
  {H -- D} -!- {J -- B} -!- {{B -- U -- T}}; {F -- E}; {G -- C}
};
\gittag
{origin/master}       % node name and text
{above=of B} % node placement
{B}          % target
\gittag
{master}       % node name and text
{right=of T} % node placement
{T}          % target
\end{tikzpicture}
\itkcaption{Additional commit on \textit{master} branch.}
\label{fig:WrongAdditionalCommitOnMasterBranch}
\end{figure}

\begin{minted}[baselinestretch=1,fontsize=\footnotesize,linenos=false,bgcolor=ltgray]{bash}
wrong$ git push origin master
\end{minted}


Rejected as \textit{non-fast-forward} (see Section~\ref{subpar:NonFastForward}
on page~\pageref{subpar:NonFastForward}).

\begin{minted}[baselinestretch=1,fontsize=\footnotesize,linenos=false,bgcolor=ltgray]{bash}
wrong$ git pull
\end{minted}

\begin{figure}
\centering
\begin{tikzpicture}
\gitDAG[grow right sep = 2em, branch down = 3em]{
  {F -- } -!- {G -- },
  {[nodes=placeholder commits] "..."} -- E -- D -- C -- B -- A,
  {H -- D} -!- {J -- B} -!- {U -- T -- M'}; {F -- E}; {G -- C}; {U -- B}; {M' -- A};
};
\gittag
{origin/master}       % node name and text
{right=of A} % node placement
{A}          % target
\gittag
{master}       % node name and text
{right=of M'} % node placement
{M'}          % target
\end{tikzpicture}
\itkcaption{Pull from upstream.}
\label{fig:PullChangesFromUpstream}
\end{figure}

\begin{minted}[baselinestretch=1,fontsize=\footnotesize,linenos=false,bgcolor=ltgray]{bash}
wrong$ git push origin master
\end{minted}


Rejected with the \textit{first parent sequence not preserved} error (see
Section~\ref{subpar:FirstParentSequenceNotPreserved} on
page~\pageref{subpar:FirstParentSequenceNotPreserved}).

The solution in this case is to recreate the merge on the proper branch.

First, create a nicely-named topic branch starting from the first-parent of the
incorrect merge.

\begin{minted}[baselinestretch=1,fontsize=\footnotesize,linenos=false,bgcolor=ltgray]{bash}
git branch topic 'master^1'
\end{minted}

\begin{figure}
\centering
\begin{tikzpicture}
\gitDAG[grow right sep = 2em, branch down = 3em]{
  {F -- } -!- {G -- },
  {[nodes=placeholder commits] "..."} -- E -- D -- C -- B -- A,
  {H -- D} -!- {J -- B} -!- {U -- T -- M'}; {F -- E}; {G -- C}; {U -- B}; {M' -- A};
};
\gittag
{origin/master}       % node name and text
{right=of A} % node placement
{A}          % target
\gittag
{master}       % node name and text
{right=of M'} % node placement
{M'}          % target
\gittag
{topic}       % node name and text
{below=of T} % node placement
{T}          % target
\end{tikzpicture}
\itkcaption{Create a \textit{topic} branch starting from the first-parent of the incorrect merge.}
\label{fig:CreateLocalBranchFromFirstParent}
\end{figure}

Then reset your local master to that from upstream.

\begin{minted}[baselinestretch=1,fontsize=\footnotesize,linenos=false,bgcolor=ltgray]{bash}
git reset --hard origin/master
\end{minted}

\begin{figure}
\centering
\begin{tikzpicture}
\gitDAG[grow right sep = 2em, branch down = 3em]{
  {F -- } -!- {G -- },
  {[nodes=placeholder commits] "..."} -- E -- D -- C -- B -- A,
  {H -- D} -!- {J -- B} -!- {U -- T}; {F -- E}; {G -- C}; {U -- B};
};
\gittag
{origin/master}       % node name and text
{above=of A} % node placement
{A}          % target
\gittag
{master}       % node name and text
{below=of A} % node placement
{A}          % target
\gittag
{topic}       % node name and text
{right=of T} % node placement
{T}          % target
\end{tikzpicture}
\itkcaption{Reset the local \textit{master} branch to upstream \textit{master}.}
\label{fig:ResetLocalMasterToUpstreamMaster}
\end{figure}

Now create the correct merge commit as described in the workflow instructions
above.

\begin{minted}[baselinestretch=1,fontsize=\footnotesize,linenos=false,bgcolor=ltgray]{bash}
git merge topic
\end{minted}

\begin{figure}
\centering
\begin{tikzpicture}
\gitDAG[grow right sep = 2em, branch down = 3em]{
  {F -- } -!- {G -- },
  {[nodes=placeholder commits] "..."} -- E -- D -- C -- B -- A -- M,
  {H -- D} -!- {J -- B} -!- {U -- T}; {F -- E}; {G -- C}; {U -- B}; {T -- M}
};
\gittag
{master}       % node name and text
{right=of M} % node placement
{M}          % target
\gittag
{topic}       % node name and text
{right=of T} % node placement
{T}          % target
\end{tikzpicture}
\itkcaption{Merge the \textit{topic} branch.}
\label{fig:MergeTopicIntoMaster4}
\end{figure}

\begin{minted}[baselinestretch=1,fontsize=\footnotesize,linenos=false,bgcolor=ltgray]{bash}
git push origin master
git branch -d topic
\end{minted}


\begin{figure}
\centering
\begin{tikzpicture}
\gitDAG[grow right sep = 2em, branch down = 3em]{
  {F -- } -!- {G -- },
  {[nodes=placeholder commits] "..."} -- E -- D -- C -- B -- A -- M,
  {H -- D} -!- {J -- B} -!- {U -- T}; {F -- E}; {G -- C}; {U -- B}; {T -- M}
};
\gittag
{origin/master}       % node name and text
{above=of M} % node placement
{M}          % target
\gittag
{master}       % node name and text
{below=of M} % node placement
{M}          % target
\end{tikzpicture}
\itkcaption{Delete the local \textit{topic} branch.}
\label{fig:DeleteLocalBranch2}
\end{figure}


\subparagraph{Topics Must Be Merged}
\label{subpar:TopicsMustBeMerged}

%TODO

%Note: I was referred to this documentation when my merge to master failed a
%pre-commit hook. The fix was to add "--no-ff" arg when merging the topic.


\subsubsection{Conflicts}
\label{subsubsec:Conflicts}

This section documents conflict resolution in a topic-based branchy workflow.

Whenever two paths of development make different changes to the same initial
content conflicts may occur when merging the branches.


\paragraph{Single Integration Branch}
\label{par:SingleIntegrationBranch}

Consider two conflicting topic branches, topic and other-topic, with the latter
already merged to \textbf{master}:

\begin{figure}
\centering
\begin{tikzpicture}
\gitDAG[grow right sep = 2em, branch down = 3em]{
  C1 -- {C2 -- C3,
    C4 -- C5,
    C6 -- C7}; {C5 -- C3}
};
\gittag
{master}       % node name and text
{right=of C3} % node placement
{C3}          % target
\gittag
{other-topic}       % node name and text
{right=of C5} % node placement
{C5}          % target
\gittag
{topic}       % node name and text
{right=of C7} % node placement
{C7}          % target
\end{tikzpicture}
\itkcaption{Conflicting topic branches.}
\label{fig:ConflictingTopicBranches1}
\end{figure}

An attempt to merge topic into master will fail with conflicts. One may use the
following approaches to resolve the situation:
\begin{itemize}
\item Merge the topic to the branch
\item Merge the branch to the topic
\end{itemize}


\subparagraph{Topic-to-Single-Branch}
\label{subpar:TopicToSingleBranch}

If one performs the merge in a local work tree it is possible to simply resolve
the conflicts and complete the merge:

\begin{figure}
\centering
\begin{tikzpicture}
\gitDAG[grow right sep = 2em, branch down = 3em]{
  C1 -- {C2 -- C3 -- C8,
    C4 -- C5,
    C6 -- C7}; {C5 -- C3}; {C7 -- C8}
};
\gittag
{master}       % node name and text
{right=of C8} % node placement
{C8}          % target
\gittag
{other-topic}       % node name and text
{right=of C5} % node placement
{C5}          % target
\gittag
{topic}       % node name and text
{right=of C7} % node placement
{C7}          % target
\end{tikzpicture}
\itkcaption{Topic-to-single branch resolution approach.}
\label{fig:TopicToSingleBranchResolutionApproach}
\end{figure}


\subparagraph{Branch-to-Topic}
\label{subpar:BranchToTopic}

Since a developer works on a topic branch locally one may simply merge the
conflicting integration branch into the topic and resolve the conflicts:

\begin{figure}
\centering
\begin{tikzpicture}
\gitDAG[grow right sep = 2em, branch down = 3em]{
  C1 -- {C2 -- C7,
    C3 -- C4,
    C5 -- C6 -- C8}; {C7 -- C4}; {C8 -- C7}
};
\gittag
{master}       % node name and text
{right=of C7} % node placement
{C7}          % target
\gittag
{other-topic}       % node name and text
{right=of C4} % node placement
{C4}          % target
\gittag
{topic}       % node name and text
{right=of C8} % node placement
{C8}          % target
\end{tikzpicture}
\itkcaption{Branch-to-Topic resolution approach.}
\label{fig:BranchToTopicResolutionApproach}
\end{figure}

In order to maintain a good shape of history one may then merge the topic into
the integration branch without allowing a \textit{fast-forward}
(\code{merge --no-ff}):

\begin{figure}
\centering
\begin{tikzpicture}
\gitDAG[grow right sep = 2em, branch down = 3em]{
  C1 -- {C2 -- C7 -- C9,
    C3 -- C4,
    C5 -- C6 -- C8}; {C7 -- C4}; {C8 -- C7}; {C9 -- C8}
};
\gittag
{master}       % node name and text
{right=of C9} % node placement
{C9}          % target
\gittag
{other-topic}       % node name and text
{right=of C4} % node placement
{C4}          % target
\gittag
{topic}       % node name and text
{right=of C8} % node placement
{C8}          % target
\end{tikzpicture}
\itkcaption{Merge disallowing fast-forward (\code{--no-ff}).}
\label{fig:MergeDisallowingFF}
\end{figure}


\paragraph{Multiple Integration Branches}
\label{par:MultipleIntegrationBranches}

In a workflow using multiple integration branches one must deal differently with
conflicting topics. Consider two conflicting topic branches, \textit{topic} and
\textit{other-topic}, with the latter already merged to \textbf{release}:

\begin{figure}
\centering
\begin{tikzpicture}
\gitDAG[grow right sep = 2em, branch down = 3em]{
  C1 -- C2 -!- {
    "" [placeholder];
    {C3 -- C4}; {C5 -- C6}, {{[nodes=placeholder commits] "..."} -- C7}};
  {C2 -- C3}; {C1 -- C5}; {C6 -- C7};
};
\gittag
{master}       % node name and text
{right=of C2} % node placement
{C2}          % target
\gittag
{topic}       % node name and text
{right=of C4} % node placement
{C4}          % target
\gittag
{other-topic}       % node name and text
{right=of C6} % node placement
{C6}          % target
\gittag
{release}       % node name and text
{right=of C7} % node placement
{C7}          % target
\end{tikzpicture}
\itkcaption{Conflicting topic branches.}
\label{fig:ConflictingTopicBranches2}
\end{figure}

An attempt to merge \textit{topic} into \textbf{release} will fail with
conflicts. One may use the following approaches to resolve the situation:
\begin{itemize}
\item Merge the topic to the branch (see Section~\ref{subpar:TopicToBranch}).
\item Merge one topic into the other (see Section~\ref{subpar:TopicToTopic}).
\item Merge both topics into a resolution topic (see
Section~\ref{subpar:ResolutionTopic}).
\end{itemize}

Note that one may not merge the branch into the topic as in the
single-integration-branch case because \textbf{release} may never be merged into
a topic.


\subparagraph{Topic-to-Branch}
\label{subpar:TopicToBranch}

If one performs the merge in a local work tree it is possible to simply resolve
the conflicts and complete the merge:

\begin{figure}
\centering
\begin{tikzpicture}
\gitDAG[grow right sep = 2em, branch down = 3em]{
  C1 -!- { C2 ,
    {C3 -- C4};
    {C5 -- C6}, {{[nodes=placeholder commits] "..."} -- C7 -- C8 -- C4}};
  {C1 -- C2}; {C1 -- C5}; {C2 -- C3}; {C6 -- C7};
};
\gittag
{master}       % node name and text
{right=of C2} % node placement
{C2}          % target
\gittag
{topic}       % node name and text
{right=of C4} % node placement
{C4}          % target
\gittag
{other-topic}       % node name and text
{right=of C6} % node placement
{C6}          % target
\gittag
{release}       % node name and text
{right=of C8} % node placement
{C8}          % target
\end{tikzpicture}
\itkcaption{Merge locally.}
\label{fig:MergeLocally}
\end{figure}

However, the topics eventually must be merged to \textbf{master}. Assume
\textit{topic} is merged first:

\begin{figure}
\centering
\begin{tikzpicture}
\gitDAG[grow right sep = 2em, branch down = 3em]{
  C1 -!- { C2 -- C9,
    {C3 -- C4};
    {C5 -- C6}, {{[nodes=placeholder commits] "..."} -- C7 -- C8 -- C4}};
  {C1 -- C2}; {C1 -- C5}; {C2 -- C3}; {C4 -- C9}; {C6 -- C7};
};
\gittag
{master}       % node name and text
{right=of C9} % node placement
{C9}          % target
\gittag
{topic}       % node name and text
{right=of C4} % node placement
{C4}          % target
\gittag
{other-topic}       % node name and text
{right=of C6} % node placement
{C6}          % target
\gittag
{release}       % node name and text
{right=of C8} % node placement
{C8}          % target
\end{tikzpicture}
\itkcaption{Merge \textit{topic}.}
\label{fig:MergeTopic1}
\end{figure}

An attempt to merge \textit{other-topic} into \textbf{master} will fail with the
same conflicts!

\begin{figure}
\centering
\begin{tikzpicture}
\gitDAG[grow right sep = 2em, branch down = 3em]{
  C1 -!- { C2 -- C9 -- ?,
    {C3 -- C4};
    {C5 -- C6}, {{[nodes=placeholder commits] "..."} -- C7 -- C8 -- C4}};
  {C1 -- C2}; {C1 -- C5}; {C2 -- C3}; {C4 -- C9}; {C6 -- C7}; {C6 -- ?}
};
\gittag
{master}       % node name and text
{right=of ?} % node placement
{C9}          % target
\gittag
{topic}       % node name and text
{right=of C4} % node placement
{C4}          % target
\gittag
{other-topic}       % node name and text
{right=of C6} % node placement
{C6}          % target
\gittag
{release}       % node name and text
{right=of C8} % node placement
{C8}          % target
\end{tikzpicture}
\itkcaption{Merge conflict when attempting to merge \textit{other-topic} into \textbf{master}.}
\label{fig:MergeConflict}
\end{figure}


The only branch that contains a resolution to these conflicts is
\textbf{release}, but that may not be merged to \textbf{master}. Therefore one
must resolve the conflicts a second time.

If the second resolution is not byte-for-byte identical to the first then the
new \textbf{master} will not merge cleanly into \textbf{release}:

\begin{figure}
\centering
\begin{tikzpicture}
\gitDAG[grow right sep = 2em, branch down = 3em]{
  C1 -!- { C2 -- C9 -- C10,
    {C3 -- C4};
    {C5 -- C6}, {{[nodes=placeholder commits] "..."} -- C7 -- C8 -- ?}};
  {C1 -- C2}; {C1 -- C5}; {C2 -- C3}; {C4 -- C8}; {C4 -- C9}; {C6 -- C7}; {C6 -- C10}; {C10 -- ?}
};
\gittag
{master}       % node name and text
{right=of C10} % node placement
{C9}          % target
\gittag
{topic}       % node name and text
{right=of C4} % node placement
{C4}          % target
\gittag
{other-topic}       % node name and text
{right=of C6} % node placement
{C6}          % target
\gittag
{release}       % node name and text
{right=of ?} % node placement
{?}          % target
\end{tikzpicture}
\itkcaption{\textbf{master} not merging cleanly into \textbf{release} if conflicts have not been resolved.}
\label{fig:UncleanMerge}
\end{figure}

Then one must resolve conflicts a third time!

This approach works with manual merging but requires care.


\subparagraph{Topic-to-Topic}
\label{subpar:TopicToTopic}

The design (see Section~\ref{subsubsec:Design} on
page~\pageref{subsubsec:Design}) of our topic-based workflow guarantees that
work is always committed on topic branches and never directly on an integration
branch. If conflicts occur while merging a topic into an integration branch it
means that the topic conflicts with \textit{another topic} that has already been
merged.

One may manually merge the conflicting \textit{other-topic} into one's own
\textit{topic} and resolve the conflicts:

\begin{figure}
\centering
\begin{tikzpicture}
\gitDAG[grow right sep = 1em, branch down = 3em]{
  C1 -- C2 -- {
    "" [placeholder];
    C3 -- C4-- C5
  },
  C6 -- C7 -!- {
    "" [placeholder];
    {[nodes=placeholder commits] "..."} -- C8
  }; {C1 -- C6}; {C7 -- C5}; {C8 -- C7}
};
\gittag
{master}       % node name and text
{right=of C2} % node placement
{C2}          % target
\gittag
{topic}       % node name and text
{right=of C5} % node placement
{C5}          % target
\gittag
{other-topic}       % node name and text
{right=of C7} % node placement
{C7}          % target
\gittag
{release}       % node name and text
{right=of C8} % node placement
{C8}          % target
\end{tikzpicture}
\itkcaption{Manually merge \textit{other-topic} into \textit{topic}.}
\label{fig:ManualMergeTopicIntoTopic}
\end{figure}

Then topic will merge cleanly into \textbf{release}:

\begin{figure}
\centering
\begin{tikzpicture}
\gitDAG[grow right sep = 1em, branch down = 3em]{
  C1 -- C2 -- {
    "" [placeholder];
    C3 -- C4-- C5
  },
  C6 -- C7 -!- {
    "" [placeholder];
    {[nodes=placeholder commits] "..."} -- C8 -- C9
  }; {C1 -- C6}; {C7 -- C5}; {C8 -- C7}; {C9 -- C5}
};
\gittag
{master}       % node name and text
{right=of C2} % node placement
{C2}          % target
\gittag
{topic}       % node name and text
{right=of C5} % node placement
{C5}          % target
\gittag
{other-topic}       % node name and text
{right=of C7} % node placement
{C7}          % target
\gittag
{release}       % node name and text
{right=of C9} % node placement
{C9}          % target
\end{tikzpicture}
\itkcaption{Merge into \textit{release}.}
\label{fig:MergeChangesIntoRelease}
\end{figure}

Later, \textit{topic} may be merged cleanly into \textbf{master} to bring in
both topics (or just \textit{topic} if \textit{other-topic} has already been
merged):

\begin{figure}
\centering
\begin{tikzpicture}
\gitDAG[grow right sep = 1em, branch down = 3em]{
  C1 -- C2 -- {
    "" [placeholder];
    C3 -- C4-- C5
  } -!- C10,
  C6 -- C7 -!- {
    "" [placeholder];
    {[nodes=placeholder commits] "..."} -- C8 -- C9
  }; {C1 -- C6}; {C7 -- C5}; {C8 -- C7}; {C9 -- C5}; {C2 -- C10}; {C5 -- C10}
};
\gittag
{master}       % node name and text
{right=of C10} % node placement
{C10}          % target
\gittag
{topic}       % node name and text
{right=of C5} % node placement
{C5}          % target
\gittag
{other-topic}       % node name and text
{right=of C7} % node placement
{C7}          % target
\gittag
{release}       % node name and text
{right=of C9} % node placement
{C9}          % target
\end{tikzpicture}
\itkcaption{Merge into \textit{master}.}
\label{fig:MergeIntoMaster}
\end{figure}

Finally, \textbf{master} may be merged cleanly into \textbf{release}:

\begin{figure}
\centering
\begin{tikzpicture}
\gitDAG[grow right sep = 1em, branch down = 3em]{
  C1 -- C2 -- {
    "" [placeholder];
    C3 -- C4-- C5
  } -!- C10,
  C6 -- C7 -!- {
    "" [placeholder];
    {[nodes=placeholder commits] "..."} -- C8 -- C9 -- C11}; {C1 -- C6}; {C7 -- C5}; {C8 -- C7}; {C9 -- C5}; {C2 -- C10}; {C5 -- C10}; {C11 -- C10}
};
\gittag
{master}       % node name and text
{right=of C10} % node placement
{C10}          % target
\gittag
{topic}       % node name and text
{right=of C5} % node placement
{C5}          % target
\gittag
{other-topic}       % node name and text
{right=of C7} % node placement
{C7}          % target
\gittag
{release}       % node name and text
{right=of C11} % node placement
{C11}          % target
\end{tikzpicture}
\itkcaption{Merge \textit{master} into \textit{release}.}
\label{fig:MergeIntoRelease1}
\end{figure}

Note that this produces an artificial topic dependency (see
Section~\ref{par:DependentTopic} on page~\pageref{par:DependentTopic})
introduced by the conflict resolution commit. See
the~\ref{subpar:ResolutionTopic} approach to avoid this problem.


\subparagraph{Resolution Topic}
\label{subpar:ResolutionTopic}

The~\ref{subpar:TopicToTopic} approach introduces an artificial topic
dependency because it asymmetrically favors one topic over another. Instead one
may use a third topic to resolve the conflicts.

One may start a new \textit{resolve/topic/other-topic} branch from
\textit{topic}, merge \textit{other-topic} into it, and resolve the conflicts:

\begin{figure}
\centering
\begin{tikzpicture}
\gitDAG[grow right sep = 1em, branch down = 3em]{
  C1 -- C2 -- {
    "" [placeholder];
    C3 -- C4 -!- {
      "" [placeholder]; C7 }
  },
  C5 -- C6 -!- {
    "" [placeholder];
    {[nodes=placeholder commits] "..."} -- C8}; {C1 -- C5}; {C7 -- C4}; {C7 -- C6}; {C8 -- C6}
};
\gittag
{master}       % node name and text
{right=of C2} % node placement
{C2}          % target
\gittag
{topic}       % node name and text
{right=of C4} % node placement
{C4}          % target
\gittag
{resolve topic/other-topic}       % node name and text
{right=of C7} % node placement
{C7}          % target
\gittag
{other-topic}       % node name and text
{right=of C6} % node placement
{C6}          % target
\gittag
{release}       % node name and text
{right=of C8} % node placement
{C8}          % target
\end{tikzpicture}
\itkcaption{Start conflict resolution branch.}
\label{fig:StartConflictResolutionBranch}
\end{figure}

The resolution topic will merge cleanly into \textbf{release} to bring in the
changes from topic through the conflict resolution commit:

\begin{figure}
\centering
\begin{tikzpicture}
\gitDAG[grow right sep = 1em, branch down = 3em]{
  C1 -- C2 -- {
    "" [placeholder];
    C3 -- C4 -!- {
      "" [placeholder]; C7 }
  },
  C5 -- C6 -!- {
    "" [placeholder];
    {[nodes=placeholder commits] "..."} -- C8 -- C9
  }; {C1 -- C5}; {C7 -- C4}; {C7 -- C6};{C8 -- C6};  {C9 -- C7}
};
\gittag
{master}       % node name and text
{right=of C2} % node placement
{C2}          % target
\gittag
{topic1}       % node name and text
{right=of C4} % node placement
{C4}          % target
\gittag
{resolve topic/other-topic}       % node name and text
{right=of C7} % node placement
{C7}          % target
\gittag
{other-topic}       % node name and text
{right=of C6} % node placement
{C6}          % target
\gittag
{release}       % node name and text
{right=of C9} % node placement
{C9}          % target
\end{tikzpicture}
\itkcaption{Merge into \textit{release}.}
\label{fig:MergeIntoRelease2}
\end{figure}

Since \textit{topic} and \textit{other-topic} are still independent either may
be merged to master first. Assume \textit{topic} is merged first:

\begin{figure}
\centering
\begin{tikzpicture}
\gitDAG[grow right sep = 1em, branch down = 3em]{
  C1 -- C2 -!- {
    "" [placeholder];
    C3 -- C4 -!- {
      "" [placeholder]; C7 }
  } -!- C10,
  C5 -- C6 -!- {
    "" [placeholder];
    {[nodes=placeholder commits] "..."} -- C8 -- C9
  }; {C1 -- C5}; {C2 -- C3}; {C2 -- C10}; {C7 -- C4}; {C4 -- C10}; {C7 -- C6}; {C8 -- C6}; {C9 -- C7}
};
\gittag
{master}       % node name and text
{right=of C10} % node placement
{C10}          % target
\gittag
{topic}       % node name and text
{right=of C4} % node placement
{C4}          % target
\gittag
{resolve topic/other-topic}       % node name and text
{right=of C7} % node placement
{C7}          % target
\gittag
{other-topic}       % node name and text
{right=of C6} % node placement
{C6}          % target
\gittag
{release}       % node name and text
{right=of C9} % node placement
{C9}          % target
\end{tikzpicture}
\itkcaption{Merge \textit{topic} branch.}
\label{fig:MergeTopic2}
\end{figure}

As in the~\ref{subpar:TopicToBranch} approach, an attempt to merge
\textit{other-topic} directly into \textbf{master} will fail with the original
conflicts but now we have a topic containing the resolution commit independent
of next. One may merge the resolution topic to \textbf{master} to bring in the
changes from \textit{other-topic} and the conflict resolution:

\begin{figure}
\centering
\begin{tikzpicture}
\gitDAG[grow right sep = 1em, branch down = 3em]{
  C1 -- C2 -!- {
    "" [placeholder];
    C3 -- C4 -!- {
      "" [placeholder]; C7 }
  } -!- C10 -- C11,
  C5 -- C6 -!- {
    "" [placeholder];
    {[nodes=placeholder commits] "..."} -- C8 -- C9
  }; {C1 -- C5}; {C2 -- C3}; {C2 -- C10}; {C7 -- C4}; {C4 -- C10}; {C7 -- C6}; {C8 -- C6}; {C9 -- C7}; {C7 -- C11}
};
\gittag
{master}       % node name and text
{right=of C11} % node placement
{C11}          % target
\gittag
{topic}       % node name and text
{right=of C4} % node placement
{C4}          % target
\gittag
{resolve topic/other-topic}       % node name and text
{right=of C7} % node placement
{C7}          % target
\gittag
{other-topic}       % node name and text
{right=of C6} % node placement
{C6}          % target
\gittag
{release}       % node name and text
{right=of C9} % node placement
{C9}          % target
\end{tikzpicture}
\itkcaption{Merge conflict resolution branch into \textit{master}.}
\label{fig:MergeConflictResolutionBranchIntoMaster}
\end{figure}

Finally, \textbf{master} may be merged cleanly into \textbf{release}:

\begin{figure}
\centering
\begin{tikzpicture}
\gitDAG[grow right sep = 1em, branch down = 3em]{
  C1 -- C2 -!- {
    "" [placeholder];
    C3 -- C4 -!- {
      "" [placeholder]; C7 }
  } -!- C10 -- C11,
  C5 -- C6 -!- {
    "" [placeholder];
    {[nodes=placeholder commits] "..."} -- C8 -- C9 -- C12
  }; {C1 -- C5}; {C2 -- C3}; {C2 -- C10}; {C7 -- C4}; {C4 -- C10}; {C7 -- C6}; {C8 -- C6}; {C9 -- C7}; {C7 -- C11}; {C12 -- C11}
};
\gittag
{master}       % node name and text
{right=of C11} % node placement
{C11}          % target
\gittag
{topic}       % node name and text
{right=of C4} % node placement
{C4}          % target
\gittag
{resolve topic/other-topic}       % node name and text
{right=of C7} % node placement
{C7}          % target
\gittag
{other-topic}       % node name and text
{right=of C6} % node placement
{C6}          % target
\gittag
{release}       % node name and text
{right=of C12} % node placement
{C12}          % target
\end{tikzpicture}
\itkcaption{Merge into \textit{release}.}
\label{fig:MergeConflictResolutionBranchIntoRelease}
\end{figure}


\subsection{Publish}
\label{subsec:Publish}


\subsubsection{Push Access}
\label{subsubsec:PushAccess}

Authorized developers may publish work directly to a \href{public.kitware.com}
repository using Git's SSH protocol.

Note that we may not grant all contributors push access to any given repository.
The distributed nature of Git allows contributors to retain authorship credit
even if they do not publish changes directly.

\paragraph{Authentication}
\label{par:Authentication}

All publishers share the \code{git@public.kitware.com} account but each uses a
unique SSH key for authentication. If you do not have a public/private SSH key
pair, generate one:

\begin{minted}[baselinestretch=1,fontsize=\footnotesize,linenos=false,bgcolor=ltgray]{bash}
ssh-keygen -C 'you@yourdomain.com'
Generating public/private rsa key pair.
Enter file in which to save the key (\$HOME/.ssh/id\_rsa):
Enter passphrase (empty for no passphrase): (use-a-passphrase!!)
Enter same passphrase again: (use-same-passphrase!!)
Your identification has been saved in \$HOME/.ssh/id\_rsa.
Your public key has been saved in \$HOME/.ssh/id\_rsa.pub.
\end{minted}

To request access, fill out the
\href{https://www.kitware.com/Admin/SendPassword.cgi}{Kitware Password} form.
Include your SSH public key, \code{id\_rsa.pub}, and a reference to someone our
administrators may contact to verify your privileges.

\subparagraph{SSH on Windows}
\label{subpar:SSHOnWindows}

If you are familiar with generating an SSH key on Linux or macOS, you can follow
the same procedure on Windows in a Git Bash prompt. There is an
\code{ssh-keygen} program installed with \code{Git for Windows} to help you set
up an SSH identity on a Windows machine. By default it puts the \code{.ssh}
directory in the \code{HOME} directory, which is typically
\code{C:\\Users\\Username}.

Alternatively, you can also set up a ``normal'' Windows command prompt shell
such that it will work with \code{Git for Windows} (see
Section~\ref{subsubsec:GitForWindows}) on page~\pageref{subsubsec:GitForWindows},
without ever invoking the Git Bash prompt if you like. If you install
\code{Git for Windows} and accept all its default options, ``git'' will not be
in the \code{PATH}. However, if you add \code{C:\\Program Files (x86)\\Git\\cmd} to
your \code{PATH}, then only the two commands \code{git} and \code{gitk} are
available to use via \code{*.cmd} script wrappers installed by
\code{Git for Windows}. Or, if you add \code{C:\\Program Files (x86)\\Git\\bin}
to your \code{PATH}, then all of the command line tools that git installs are
available.

The full
{PuTTY}\footnote{\url{https://www.chiark.greenend.org.uk/~sgtatham/putty/}}
suite of tools includes an application called \code{PuTTYgen}. If you
already have a private key created with \code{PuTTYgen}, you may export it to an
OpenSSH identity file. Open the key using \code{PuTTYgen} and choose
\textit{Conversions > Export OpenSSH key} from the menu bar. That will allow you
to save an \code{id\_rsa} file for use in the \code{.ssh} directory. You can also
copy and paste the public key portion of the key from the \code{PuTTYgen} text
field to save into an \code{id\_rsa.pub} file if you like. Or email it to whoever
needs the public side of your key pair.

If you routinely set up your own command prompt environment on Windows, using
\code{Git for Windows} from that environment is a cinch: just add the full path
to either \code{Git\\cmd} or \code{Git\\bin} to your \code{PATH}. (Or, write your
own \code{git.cmd} wrapper that is in your \code{PATH} that simply calls the
\code{git.cmd} installed with msysGit.) And make sure you have a \code{HOME}
environment variable that points to the place where the \code{.ssh}
directory is.


\subparagraph{AuthenticationTest}
\label{subpar:AuthenticationTest}

When your SSH public key has been installed for \code{git@public.kitware.com},
you may test your SSH key setup by running

\begin{minted}[baselinestretch=1,fontsize=\footnotesize,linenos=false,bgcolor=ltgray]{bash}
ssh git@public.kitware.com info
\end{minted}

If your key is correctly configured you should see a message reporting your
email address followed by a list of access permissions. If you get something
like \textit{Permission denied} then add \code{-v} options to your ssh command
line to diagnose the problem:

\begin{minted}[baselinestretch=1,fontsize=\footnotesize,linenos=false,bgcolor=ltgray]{bash}
ssh -v git@public.kitware.com info
\end{minted}

Do not attempt to \code{git push} until the ssh-only test succeeds.


\paragraph{Pushing}
\label{subpar:Pushing}

Git automatically configures a new clone to refer to its \textit{origin} through
a \textit{remote} called \code{origin}. Initially one may \code{fetch} or
\code{pull} changes from origin, but may not \code{push} changes to it.

In order to publish new commits in a \href{public.kitware.com} repository,
developers must configure a push URL for the origin. Use \code{git config} to
specify an SSH-protocol URL:

\begin{minted}[baselinestretch=1,fontsize=\footnotesize,linenos=false,bgcolor=ltgray]{bash}
git config remote.origin.pushurl git@public.kitware.com:repo.git
\end{minted}

The actual URL will vary from project to project. (Note that \code{pushurl}
requires Git >= 1.6.4. Use just \code{url} for Git < 1.6.4.)

Failing to do so with result in the error message \textit{fatal: The
remote end hung up unexpectedly}.

Once your push URL is configured and your key is installed for
\code{git@public.kitware.com} then you can try pushing changes. Note that many
repositories use an \code{update} hook to check commit as documented in
Section~\ref{subsec:Hooks} on page~\pageref{subsec:Hooks}.


\subsubsection{Patches}
\label{subsubsec:Patches}

Git allows anyone to be a first-class developer on any published project. One
can clone a public repository, commit locally, and publish these commits for
inclusion upstream. One method of sending commits upstream is to supply them as
patches.

See these links for more help:
\begin{itemize}
\item Pro Git Book, Chapter 5: Distributed Git\footnote{\url{https://git-scm.com/book/en/v2}}
\item Everyday Git: Integrator\footnote{\url{https://git-scm.com/docs/giteveryday}}
\end{itemize}


\paragraph{Creating Patches}
\label{par:CreatingPatches}

Construct your commits on a local topic branch, typically started from the
\textit{upstream} \textbf{master}:

\begin{minted}[baselinestretch=1,fontsize=\footnotesize,linenos=false,bgcolor=ltgray]{bash}
git checkout -b my-cool-feature origin/master
edit files
git add -- files
git commit
\end{minted}

Begin each commit message with a short one-line summary of its change, suitable
for use as an email subject line. Then leave a blank line and describe the
change in detail as one might write in an email body.

When the patch(es) are ready for publication to upstream developers, use the
\code{git format-patch} command to construct the patch files:

\begin{minted}[baselinestretch=1,fontsize=\footnotesize,linenos=false,bgcolor=ltgray]{bash}
git format-patch -M origin/master
\end{minted}

Git will write out one patch file per commit. Each patch file is formatted like
a raw email message and includes enough information to reconstruct the commit
message and author.


\paragraph{Sending Patches}
\label{par:SendingPatches}

The patch files created in the preceding step will be named with the form

\begin{minted}[baselinestretch=1,fontsize=\footnotesize,linenos=false,bgcolor=ltgray]{bash}
NNNN-Subject-line-of-commit-message.patch
\end{minted}

where \code{NNNN} is an index for the patch within the series. These files may
be attached in bug trackers or attached to email messages.

A patch series may also be sent directly as email. Use
\code{git config --global} to set \code{sendemail.*} configuration entries that
tell Git how to send email from your computer (one-time setup per user per
machine). Then use the \code{git send-email} command:

\begin{minted}[baselinestretch=1,fontsize=\footnotesize,linenos=false,bgcolor=ltgray]{bash}
git send-email *.patch --to='Some One <someone@somewhere.com>' --cc='Someone Else <someoneelse@somewhereelse.com>'
\end{minted}


\paragraph{Applying Patches}
\label{par:ApplyingPatches}

One may receive patches as attachments in a bug tracker or as attachments to
email messages. Save these files to your local disk. One may also receive
patches inlined in email messages. In this case, save the whole message to your
local disk (typically as \code{.eml} files). (If your local mail client uses
\code{maildir} format mailboxes each message is already its own file.)

Create a local topic branch on which to replay the patch series:

\begin{minted}[baselinestretch=1,fontsize=\footnotesize,linenos=false,bgcolor=ltgray]{bash}
git checkout -b cool-feature origin/master
\end{minted}

Now use \code{git am} to apply the patch series as local commits:

\begin{minted}[baselinestretch=1,fontsize=\footnotesize,linenos=false,bgcolor=ltgray]{bash}
git am --whitespace=fix /path/to/*.patch
\end{minted}

Review the changes using

\begin{minted}[baselinestretch=1,fontsize=\footnotesize,linenos=false,bgcolor=ltgray]{bash}
git log -p origin/master..
\end{minted}

or the method of your choice. Note that the author of each commit is the
contributor rather than yourself. Build, test, and publish the changes normally.

If the \code{git am} command fails with a message like

\begin{minted}[baselinestretch=1,fontsize=\footnotesize,linenos=false,bgcolor=ltgray]{bash}
Patch format detection failed.
\end{minted}

this means that the patch was not generated with \code{git format-patch} and
transmitted correctly. Either ask the contributor to try again using the above
patch creation instructions, or apply each patch separately using
\code{git apply}:

\begin{minted}[baselinestretch=1,fontsize=\footnotesize,linenos=false,bgcolor=ltgray]{bash}
git apply --whitespace=fix /path/to/0001-First-change.patch
git commit --author='Contributor Name <contributor@theirdomain.com>'
\end{minted}


\subsection{Hooks}
\label{subsec:Hooks}

\subsubsection{Setup}
\label{subsubsec:Setup}

The \code{git commit} command creates local commits. A separate \code{git push}
step is needed to publish commits to a \href{public.kitware.com} repository. The
\href{public.kitware.com} server enforces some rules (see
Section~\ref{par:Update} on page~\pageref{par:Update}) on the commits it
accepts and will reject non-conforming commits. In order to push rejected
commits, one must edit history locally to repair them before publishing.

Since it is possible to create many commits locally and push them all at once,
we provide local Git \textit{hooks} to help developers keep their individual
commits clean. Git provides no way to enable such hooks by default, giving
developers maximum control over their local repositories. We recommend enabling
our hooks manually in all clones.

Git looks for hooks in the \code{.git/hooks} directory within the work tree of a
local repository. Create a new local repository in this directory to manage the
hooks:

\begin{minted}[baselinestretch=1,fontsize=\footnotesize,linenos=false,bgcolor=ltgray]{bash}
cd .git/hooks
git init
cd ../..
\end{minted}

Choose one of the following methods to install or update the hooks. The hooks
will then run in the outer repository to enforce some rules on commits.


\paragraph{Local Pull}
\label{par:LocalPull}

Many \href{public.kitware.com} repositories provide a \code{hooks} branch. It
will have already been fetched into your local clone. Pull it from there:

\begin{minted}[baselinestretch=1,fontsize=\footnotesize,linenos=false,bgcolor=ltgray]{bash}
git fetch origin
cd .git/hooks
git pull .. remotes/origin/hooks
cd ../..
\end{minted}


\paragraph{Direct Pull}
\label{par:DirectPull}

If you did not clone from a \href{public.kitware.com} repository you may not
have a \code{hooks} branch. Pull it from \href{public.kitware.com}:

\begin{minted}[baselinestretch=1,fontsize=\footnotesize,linenos=false,bgcolor=ltgray]{bash}
cd .git/hooks
git pull git://public.kitware.com/<repo>.git hooks
cd ../..
\end{minted}

where \code{<repo>.git} is the name of your project repository.


\subsubsection{Local}
\label{subsubsec:Local}

The above sequences maintain the following local \code{hooks} in your
repository. See Git help on
\href{githooks}{https://git-scm.com/docs/githooks} for more details.


\paragraph{pre-commit}
\label{par:pre-commit}

This runs during \code{git commit}. It checks identity and content of changes:
\begin{itemize}
\item Git \code{user.name} and \code{user.email} are set to something
reasonable.
\item Git's standard whitespace checks (see help on \code{git diff --check}).
\item The staged changes do not introduce any leading tabs in source files (we
i.ndent with spaces)
\item File modes look reasonable (no executable \code{.cxx} files, scripts with
shebang lines are executable).
\item File size is not too large (do not commit big data files; prints limit and
instructions on rejection).
\item Submodule updates are staged alone or explicitly allowed (prints
instructions on rejection).
\end{itemize}

One of Git's standard whitespace checks is to reject trailing whitespace on
lines that were added or modified. Many people consider extra space characters
at the end of a line to be an unprofessional style (including Git's own
developers), but some don ot care. Text editors typically have a mode to
highlight trailing whitespace:

\begin{itemize}
\item \textbf{Emacs}
\begin{minted}[baselinestretch=1,fontsize=\footnotesize,linenos=false,bgcolor=ltgray]{elisp}
(custom-set-variables '(show-trailing-whitespace t))
\end{minted}

\item \textbf{Vim}
\begin{minted}[baselinestretch=1,fontsize=\footnotesize,linenos=false,bgcolor=ltgray]{vim}
:highlight ExtraWhitespace ctermbg=red guibg=red
:match ExtraWhitespace /\s\+\$/
\end{minted}

\item \textbf{Microsoft Visual Studio}
To toggle viewing of white space characters, with a source file document active,
choose the menu item:

\begin{minted}[baselinestretch=1,fontsize=\footnotesize,linenos=false,bgcolor=ltgray]{bash}
Edit > Advanced > View White Space

(2-stroke keyboard shortcut: Ctrl+R, Ctrl+W)
\end{minted}

\item \textbf{Notepad++ (v7.5.1)}
To eliminate trailing white space, choose the menu item:

\begin{minted}[baselinestretch=1,fontsize=\footnotesize,linenos=false,bgcolor=ltgray]{bash}
Edit > Blank Operations > Trim Trailing Space
\end{minted}

To toggle viewing of white space characters, choose from the
menu items:

\begin{minted}[baselinestretch=1,fontsize=\footnotesize,linenos=false,bgcolor=ltgray]{bash}
View > Show Symbol > (multiple items, choose one...)
\end{minted}

\end{itemize}

If you really don't want to keep your code clean of trailing whitespace, you can
disable this part of Git's checks locally:

\begin{minted}[baselinestretch=1,fontsize=\footnotesize,linenos=false,bgcolor=ltgray]{bash}
git config core.whitespace "-blank-at-eol"
\end{minted}


\paragraph{commit-msg}
\label{par:commit-msg}

This runs during \code{git commit}. It checks the commit message format:
\begin{itemize}
\item The first line must be between 8 and 78 characters long. If you were
writing an email to describe the change, this would be the
\textit{Subject line}. Use the pre-defined prefixes (e.g. \code{ENH:} or
\code{BUG:}); they are valuable to allow the user get a fast understanding of
the change.
\item The first line must not have leading or trailing whitespace.
\item The second line must be blank, if present.
\item The third line and below may be free-form. Usually, a summary of the
commit changes is written. Try to keep paragraph text formatted in 72
columns (this is not enforced).
\end{itemize}

GUI and text-based tools that help view history typically use the first line
(\textit{Subject line}) from the commit message to give a one-line summary of
each commit. This allows a medium-level view of history, but works well only if
developers write good \textit{Subject lines} for their commits.

Examples of improper commit messages:

\begin{minted}[baselinestretch=1,fontsize=\footnotesize,linenos=false,bgcolor=ltgray]{bash}
Fixed
\end{minted}

This is too short and not informative at all.

\begin{minted}[baselinestretch=1,fontsize=\footnotesize,linenos=false,bgcolor=ltgray]{bash}
I did a really complicated change and I am trying to describe the entire thing
with a big message entered on the command line.
\end{minted}

Some good tips on why good commit messages matter can be found in the post
\textit{How to Write a Git Commit Message}\footnote{\url{https://chris.beams.io/posts/git-commit/}}.

Many CVS users develop the habit of using the \code{-m} commit option to specify
the whole message on the command line. This is probably because in CVS it is
hard to abort a commit if it already brought up the message editor. In Git this
is trivial. Just leave the message blank and the whole commit will be aborted.
Furthermore, since commits are not published automatically it is easy to allow
the commit to complete and then fix it with \code{git commit --amend}.


\subsubsection{Server}
\label{subsubsec:Server}

Many \url{public.kitware.com} repositories have server-side \code{hooks}.


\paragraph{Update}
\label{par:Update}

The update hook runs when someone tries to update a ref on the server by
pushing. The hook checks all commits included in the push:
\begin{itemize}
\item Commit author and committer must have valid email address domains (DNS
lookup succeeds).
\item Commit message does not start with \code{WIP:}. (Use the prefix locally
for work-in-progress that must be rewritten before publishing.)
\item Changes to paths updated by robots (such as \code{Utilities/kwsys}) are
not allowed.
\item No ``large'' blobs may be pushed. The limit is set on a per-repository
basis and is typically 1 MB or so.
\item No CRLF newlines may be added in the repository (see \code{core.autocrlf}
in \code{git help config}).
\item Submodules (if any) must be pushed before the references to them are
pushed.
\end{itemize}

\subsection{TipsAndTricks}
\label{subsec:TipsAndTricks}

\subsubsection{Editor support}
\label{subsubsec:EditorSupport}

Emacs users: if you put this line in your \code{.emacs} file:

\begin{minted}[baselinestretch=1,fontsize=\footnotesize,linenos=false,bgcolor=ltgray]{elisp}
(setq auto-mode-alist (cons '("COMMIT\_EDITMSG\$" . auto-fill-mode) auto-mode-alist))
\end{minted}

Git will automatically wrap your commit messages, which is what good Git
etiquette requires.


\subsubsection{Shell Customization}
\label{subsubsec:ShellCustomization}

\paragraph{Bash Completion}
\label{par:BashCompletion}

Bash users: Git comes with a set of completion options that are very useful. The
location of the file varies depending on your system:

\begin{minted}[baselinestretch=1,fontsize=\footnotesize,linenos=false,bgcolor=ltgray]{bash}
# Mac with git installed by Mac Ports
source /opt/local/share/doc/git-core/contrib/completion/git-completion.bash
# Linux
source /usr/share/bash-completion/git
# Linux debian/gentoo
source /etc/bash\_completion.d/git
\end{minted}


\paragraph{Bash Prompt}
\label{par:BashPrompt}

If you are using the bash shell, you can customize the prompt to show which Git
branch is active. Here are the commands for your \code{~/.bashrc} file:

\begin{minted}[baselinestretch=1,fontsize=\footnotesize,linenos=false,bgcolor=ltgray]{bash}
# Use the appropriate path from the above section
source /etc/bash\_completion.d/git
export GIT\_PS1\_SHOWDIRTYSTATE=1
export GIT\_PS1\_SHOWUNTRACKEDFILES=1
export GIT\_PS1\_SHOWUPSTREAM="verbose"
export PS1="[\[\e[01;34m\]\W\[\e[31m\]\$(\_\_git\_ps1 " (%s)")\[\e[00m\]]\[\e[00m\]
\end{minted}

For more information on the options, see the comments in the top of the bash
completion script.

\paragraph{Renaming}
\label{par:Renaming}

Git does not explicitly track renames. The command

\begin{minted}[baselinestretch=1,fontsize=\footnotesize,linenos=false,bgcolor=ltgray]{bash}
git mv old new
\end{minted}

is equivalent to

\begin{minted}[baselinestretch=1,fontsize=\footnotesize,linenos=false,bgcolor=ltgray]{bash}
mv old new
git add new
git rm old
\end{minted}

Neither approach records the rename outright. However, Git's philosophy is
``dumb add, smart view''. It uses heuristics to detect renames when viewing
history after-the-fact. It even works when the content of a renamed file changes
slightly.

In order to help Git efficiently detect the rename, it is important to remove
the old file and add the new one in one commit, perhaps by using \code{git mv}
or the above 3-step procedure. If the new file were added in one commit and the
old file removed in the next, Git would report this as a copy followed by a
removal. Its copy-detection heuristics are more computationally intensive and
must be explicitly enabled with the \code{-C} option to relevant operations
(such as \code{git blame}).
