\chapter*{Abstract}
\noindent
The National Library of Medicine Insight Segmentation and Registration Toolkit,
shortened as the Insight Toolkit \href{https://itk.org}{(ITK)}, is an
open-source software toolkit for performing registration and
segmentation. \emph{Segmentation} is the process of identifying and
classifying data found in a digitally sampled
representation. Typically the sampled representation is an image
acquired from such medical instrumentation as CT or MRI
scanners. \emph{Registration} is the task of aligning or developing
correspondences between data. For example, in the medical environment,
a CT scan may be aligned with a MRI scan in order to combine the
information contained in both.

ITK is a cross-platform software. It uses a build environment known as
\href{https://cmake.org}{CMake} to manage platform-specific project
generation and compilation process in a platform-independent way. ITK is
implemented in C++. ITK's implementation style employs generic programming,
which involves the use of templates to generate, at compile-time, code that can
be applied \emph{generically} to any class or data-type that supports the
operations used by the template. The use of C++ templating means that the code
is highly efficient and many issues are discovered at compile-time, rather than
at run-time during program execution. It also means that many of ITK's
algorithms can be applied to arbitrary spatial dimensions and pixel types.

An automated wrapping system integrated with ITK generates an interface between
C++ and a high-level programming language \href{https://www.python.org}{Python}.
This enables rapid prototyping and faster exploration of ideas by shortening the
edit-compile-execute cycle. In addition to automated
wrapping, the \href{https://www.itk.org/Wiki/SimpleITK}{SimpleITK} project
provides a streamlined interface to ITK that is available for C++, Python, Java,
CSharp, R, Tcl and Ruby.

Developers from around the world can use, debug, maintain, and extend the
software because ITK is an open-source project. ITK uses a
model of software development known as Extreme
Programming. Extreme Programming collapses the usual software development
methodology into a simultaneous iterative process of
design-implement-test-release. The key features of Extreme Programming
are communication and testing. Communication among the members of the
ITK community is what helps manage the rapid evolution of the
software. Testing is what keeps the software stable. An
extensive testing process supported by the system known as
\href{https://open.cdash.org/index.php?project=Insight}{CDash}
measures the quality of ITK code on a daily basis. The ITK Testing Dashboard is
updated continuously, reflecting the quality of the code at any moment.

The most recent version of this document is available online at
\url{https://itk.org/ItkSoftwareGuide.pdf}.
