\chapter{Filtering}

This chapter introduces the most commonly used filters in the toolkit.  Most of
these filters are intended to process images. They will accept one or more
images as input and wil produce one or more images as output. Insight is based
on a data pipeline architecture in which the output of one filter is passed as
input to another filter.


\section{Thresholding}
\label{sec:ThresholdingFiltering}

\subsection{Binary Thresholding}
\label{sec:BinaryThresholdingImageFilter}

\input BinaryThresholdImageFilter.tex

\subsection{Thresholding}
\label{sec:ThresholdingImageFilter}

\input ThresholdImageFilter.tex



\section{Casting}
\label{sec:CastingFiltering}

\input CastingImageFilters.tex


\section{Gradients}
\label{sec:GradientFiltering}

Computation of gradients is a fairly common operation in image processing. The
term is sometimes loosely used to refer the gradient vectors or the magnitude
of this gradient. Insight filters attempt to reduce this ambiguity by including
the \emph{magnitude} term when appropriate. Insight provides filters for
computing both the image of gradient vectors and the image of magnitudes.

\subsection{Gradient Magnitude}
\label{sec:GradientMagnitudeImageFilter}

\input GradientMagnitudeImageFilter.tex



\section{Neighborhood Filters}
\label{sec:NeighborhoodFilters}

\subsection{Median Filter}
\label{sec:MedianFilter}

\input MedianImageFilter.tex


\subsection{Mathematical Morphology}
\label{sec:MathematicalMorphology}

\input MathematicalMorphologyFilters.tex
