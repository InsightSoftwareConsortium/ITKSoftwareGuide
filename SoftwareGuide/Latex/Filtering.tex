\chapter{Filtering}

This chapter introduces the most commonly used filters in the toolkit.  Most of
these filters are intended to process images. They will accept one or more
images as input and will produce one or more images as output. Insight is based
on a data pipeline architecture in which the output of one filter is passed as
input to another filter.


\section{Thresholding}
%\ifitkFullVersion
\label{sec:ThresholdingFiltering}
%\fi

\subsection{Binary Thresholding}
\label{sec:BinaryThresholdingImageFilter}

%\ifitkFullVersion
\input{BinaryThresholdImageFilter.tex}
%\fi

\subsection{Thresholding}
\label{sec:ThresholdingImageFilter}

%\ifitkFullVersion
\input{ThresholdImageFilter.tex}
%\fi



\section{Casting}
\label{sec:CastingImageFilters}

The filters discussed in this section perform pixel-wise intensity mappings.
This is usually desired as a parallel action to image pixel-type casting since
the input and output pixel-types have in general different dynamic ranges.

\subsection{Linear Mappings}
\label{sec:IntensityLinearMapping}

%\ifitkFullVersion
\input{CastingImageFilters.tex}
%\fi

\subsection{Non Linear Mappings}
\label{sec:IntensityNonLinearMapping}

The following filter can be seen as a variant of the casting filters. Its main
difference is the use of a smooth and continuous transtion function of
non-linear form.

%\ifitkFullVersion
\input{SigmoidImageFilter.tex}
%\fi
  

\section{Gradients}
\label{sec:GradientFiltering}

Computation of gradients is a fairly common operation in image processing. The
term is sometimes loosely used to refer the gradient vectors or the magnitude
of this gradient. Insight filters attempt to reduce this ambiguity by including
the \emph{magnitude} term when appropriate. Insight provides filters for
computing both the image of gradient vectors and the image of magnitudes.

\subsection{Gradient Magnitude}
\label{sec:GradientMagnitudeImageFilter}

%\ifitkFullVersion
\input{GradientMagnitudeImageFilter.tex}
%\fi

\subsection{Gradient Magnitude With Smoothing}
\label{sec:GradientMagnitudeRecursiveGaussianImageFilter}

%\ifitkFullVersion
\input{GradientMagnitudeRecursiveGaussianImageFilter.tex}
%\fi


\subsection{Derivative Without Smoothing}
\label{sec:DerivativeImageFilter}

%\ifitkFullVersion
\input{DerivativeImageFilter.tex}
%\fi




\section{Neighborhood Filters}
\label{sec:NeighborhoodFilters}

The concept locality is frequently encountered in image processing on the form
of filters that compute every output pixel using information from a reduced
region on the neighborhood of the input pixel. The classical form of this
filters are the $3 \times 3$ filters in 2D images. Convolution masks based on
these neighborhoods could perform diverse tasks ranging from noise reduction,
to differential operations, to mathematical morphology.

The Insight toolkit implements an elegant approach for the computation of these
family of filters. The input image is visited by a special iterator called the
\code{NeighborhoodIterator}. This iterator is capable of moving over all
the pixels in an image and for each position it can address the pixels in a
local neighborhood. Operators can be defined in order to specify what
algorithmic operation must be performed on the neighborhood of the input pixel
to compute the value of the output pixel. The following section describes some
of the commonly used filters that take advantage of this construction.  

\subsection{Mean Filter}
\label{sec:MeanFilter}

%\ifitkFullVersion
\input{MeanImageFilter.tex}
%\fi

\subsection{Median Filter}
\label{sec:MedianFilter}

%\ifitkFullVersion
\input{MedianImageFilter.tex}
%\fi


\subsection{Mathematical Morphology}
\label{sec:MathematicalMorphology}

Mathematical morphology has proved to be a powerful resource for image
processing and analysis \cite{Serra1982}. Insight implements mathematical
morphology filters using the approach of \code{NeighborhoodIterator}s and
\code{NeighborhoodOperator}s. Two basic flavors of filters are available in the
toolkit, the ones that operate on binary images and the ones that operate on
grayscale images. 

\subsubsection{Binary Filters}
\label{sec:MathematicalMorphologyBinaryFilters}

%\ifitkFullVersion
\input{MathematicalMorphologyBinaryFilters.tex}
%\fi


\subsubsection{Grayscale Filters}
\label{sec:MathematicalMorphologyGrayscaleFilters}

%\ifitkFullVersion
\input{MathematicalMorphologyGrayscaleFilters.tex}
%\fi




\section{Smoothing Filters}
\label{sec:SmoothingFilters}

Real image data has a level of uncertainty that is manifested on the
variability of measures assigned to pixels. This is usually interpreted as
noise and considered an undesirable component of the image data. This section
describes several methods that can be applied to reduce noise on images.

\subsection{Blurring}
\label{sec:BlurringFilters}

Blurring is the traditional approach for removing noise from images. It 
usually implemented in the form of a convolution with a kernel. The effect of
this operation on the image spectrum is to attenuate high spatial frequencies.
Different kernels attenuate frequencies in different ways. One of the most
commonly used kernels is the Gaussian. Two implementations of Gaussian
smoothing are available in the toolkit. The first one is based on a traditional
convolution while the other is based on the application of IIR filters that
approximate the convolution with a Gaussian \cite{Deriche1990,Deriche1993}. 

\subsubsection{Discrete Gaussian}
\label{sec:DiscreteGaussianImageFilter}

%\ifitkFullVersion
\input{DiscreteGaussianImageFilter.tex}
%\fi


\subsubsection{Binomial Blurring}
\label{sec:BinomialBlurImageFilter}

%\ifitkFullVersion
\input{BinomialBlurImageFilter.tex}
%\fi

\subsubsection{Recursive Gaussian IIR}
\label{sec:RecursiveGaussianImageFilter}

%\ifitkFullVersion
\input{SmoothingRecursiveGaussianImageFilter.tex}
%\fi



\subsection{Edge Preserving Smoothing}
\label{sec:EdgePreservingSmoothingFilters}

\subsubsection{Introduction to Anisotropic Diffusion}
\label{sec:IntroductionAnisotropicDiffusion}
%\ifitkFullVersion
\input{AnisotropicDiffusionFiltering.tex}
%\fi


\subsubsection{Gradient Anisotropic Diffusion}
\label{sec:GradientAnisotropicDiffusionImageFilter}

%\ifitkFullVersion
\input{GradientAnisotropicDiffusionImageFilter.tex}
%\fi



\subsubsection{Curvature Anisotropic Diffusion}
\label{sec:CurvatureAnisotropicDiffusionImageFilter}

%\ifitkFullVersion
\input{CurvatureAnisotropicDiffusionImageFilter.tex}
%\fi

\subsubsection{Curvature Flow}
\label{sec:CurvatureFlowImageFilter}

%\ifitkFullVersion
\input{CurvatureFlowImageFilter.tex}
%\fi

\subsubsection{MinMaxCurvature Flow}
\label{sec:MinMaxCurvatureFlowImageFilter}

%\ifitkFullVersion
\input{MinMaxCurvatureFlowImageFilter.tex}
%\fi


\subsubsection{Bilateral Filter}
\label{sec:BilateralImageFilter}

%\ifitkFullVersion
\input{BilateralImageFilter.tex}
%\fi



\subsection{Edge Preserving Smoothing in Vector Images}
\label{sec:VectorAnisotropicDiffusion}

Anisotropic diffusion can also be applied to images whose pixels are vectors.
In this case the diffusion is computed independently for each vector component.
The following classes implement versions of anisotropic diffusion on vector images.


\subsubsection{Vector Gradient Anisotropic Diffusion}
\label{sec:VectorGradientAnisotropicDiffusionImageFilter}

%\ifitkFullVersion
\input{VectorGradientAnisotropicDiffusionImageFilter.tex}
%\fi

\subsubsection{Vector Curvature Anisotropic Diffusion}
\label{sec:VectorCurvatureAnisotropicDiffusionImageFilter}

%\ifitkFullVersion
\input{VectorCurvatureAnisotropicDiffusionImageFilter.tex}
%\fi



\subsection{Edge Preserving Smoothing in Color Images}
\label{sec:ColorAnisotropicDiffusion}

\subsubsection{Gradient Anisotropic Diffusion}
\label{sec:ColorGradientAnisotropicDiffusion}

%\ifitkFullVersion
\input{RGBGradientAnisotropicDiffusionImageFilter.tex}
%\fi

\subsubsection{Curvature Anisotropic Diffusion}
\label{sec:ColorCurvatureAnisotropicDiffusion}

%\ifitkFullVersion
\input{RGBCurvatureAnisotropicDiffusionImageFilter.tex}
%\fi



\section{Distance Map}
\label{sec:DistanceMap}

%\ifitkFullVersion
\input{DanielssonDistanceMapImageFilter.tex}
%\fi



\section{Geometrical Transformations}
\label{sec:GeometricalTransformationFilters}

\subsection{Resample Image Filter}
\label{sec:ResampleImageFilter}

\subsubsection{Introduction}

%\ifitkFullVersion
\input{ResampleImageFilter.tex}
%\fi

\subsubsection{Importance of Spacing and Origin}
%\ifitkFullVersion
\input{ResampleImageFilter2.tex}
%\fi

\subsubsection{A full example}
%\ifitkFullVersion
\input{ResampleImageFilter3.tex}
%\fi

\subsubsection{Rotating an Image}
%\ifitkFullVersion
\input{ResampleImageFilter4.tex}
%\fi

\subsubsection{Rotating and Scaling an Image}
%\ifitkFullVersion
\input{ResampleImageFilter5.tex}
%\fi


