
\chapter{Segmentation}

Segmentation of medical images is a challenging task. Myriads of different
methods have been proposed and implemented in recent years. In spite of the
huge effort invested on this problem, there is no a single approach that could
solve in general the problem of segmentation for the large variety of image
modalities existing today.

The most effective segmentation algorithms are obtained by carefully
customizing combination of components. The parameters of these components are
tunned for the characteristics of the image modality used as input and the
features of the anatomical structure to be segmented. 

The Insight toolkit provides a basic set of algorithms that can be used to
develop and customize a full segmentation application. Some of the most
commonly used segmentation components are described in the following sections.


\section{Region Growing}

Region growing algorithms have proved to be a very effective approach for image
segmentation. The basic concept of a region growing algorithm is to start from
a seed region that is considered to belong to the object to be segmented. The
neighbor pixels to this initial region are evaluated to determine if they could
also be considered part of the object, in which case, they are added to the
region. When some of the neighbor pixels are included in the region, other
pixels become new neighbors and hence become candidates to be evaluated and
eventually included in the region. Region growing algorithms vary depending on
the criteria used to decide whether a pixel should be included in the region
or not, the type of neighbor connectivity used on the image grid and the
strategy used for visiting the neighbor pixels.

Several implementations of region growing are available in the
Insight toolkit. This section describes some of the most commonly used.

\subsection{Confidence Connected}
\label{sec:ConfidenceConnected}
\input{ConfidenceConnected.tex}

\subsection{Threshold Connected}

\section{Segmentation Based on Watersheds}
\label{sec:WatershedSegmentation}
%\input WatershedSegmentation.tex



%% \section{Level Sets Segmentation}
%% \label{sec:LevelSetsSegmentation}

%% \subsection{Threshold Level Set Segmentation}
%% \subsection{Fast Marching Segmentation}
%% \subsection{Shape Detection Segmentation}
%% \subsection{Geodesic Contours Segmentation}


